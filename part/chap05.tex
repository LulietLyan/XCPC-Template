\chapter{Basic DP}\label{chap:BasicDP}
\section{Knapsack Problem}
\subsection{01\ Knapsack}
\begin{lstlisting}
const int N = 1010;
int n, m, v[N], w[N], f[N];
int main()
{
  cin >> n >> m;
  for (int i = 1; i <= n; i++)
      cin >> v[i] >> w[i];
  for (int i = 1; i <= n; i++)
      for (int j = m; j >= v[i]; j++)
          f[j] = max(f[j], f[j - v[i]] + w[i]);
  cout << f[m];
}
\end{lstlisting}
\subsection{Complete Knapsack}
\begin{lstlisting}
const int N = 1010;
int n, m, v[N], w[N], f[N];
int main()
{
  cin >> n >> m;
  for (int i = 1; i <= n; i++)
      cin >> v[i] >> w[i];
  for (int i = 1; i <= n; i++)
      for (int j = v[i]; j <= m; j++)
          f[j] = max(f[j], f[j - v[i]] + w[i]);
  cout << f[m];
}
\end{lstlisting}
\subsection{Mutiple Knapsack}
\begin{lstlisting}
const int N = 25010;
int n, m, f[N];
int main()
{
  cin >> n >> m;
  for (int i = 0; i < n; i++)
  {
    int v, w, s;
    cin >> v >> w >> s;
    for (int k = 1; k <= s; k *= 2)
    {
      for (int j = m; j >= k * v; j--)
        f[j] = max(f[j], f[j - k * v] + k * w);
      s -= k;
    }
    if (s)
      for (int j = m; j >= s * v; j--)
        f[j] = max(f[j], f[j - s * v] + s * w);
  }
  cout << f[m];
}
\end{lstlisting}
\subsection{Grouped Knapsack}
\begin{lstlisting}
const int N = 120;
int n, m, s[N], v[N][N], w[N][N], f[N];
int main()
{
  cin >> n >> m;
  for (int i = 1; i <= n; i++)
  {
      cin >> s[i];
      for (int j = 1; j <= s[i]; j++)
          cin >> v[i][j] >> w[i][j];
  }
  for (int i = 1; i <= n; i++)
      for (int j = m; j >= 0; j--)
          for (int k = 1; k <= s[i]; k++)
              if (v[i][k] <= j)
                  f[j] = max(f[j], f[j - v[i][k]] + w[i][k]);
  cout << f[m];
}
\end{lstlisting}
\section{Linear DP}
\subsection{LIS}
Here is an $O(n^2)$ solution:
\begin{lstlisting}
const int N = 1010;
int n, a[N], f[N];
int main()
{
  cin >> n;
  for (int i = 1; i <= n; i++)
      cin >> a[i];
  for (int i = 1; i <= n; i++)
  {
      f[i] = 1;
      for (int j = 1; j < i; j++)
          if (a[j] < a[i])
              f[i] = max(f[i], f[j] + 1);
  }
  int res = 0;
  for (int i = 1; i <= n; i++)
      res = max(res, f[i]);
  cout << res;
}
\end{lstlisting}
Another is an $O(nlogn)$ solution:
\begin{lstlisting}
const int N = 100010;
int n, a[N], q[N];
int main()
{
  cin >> n;
  for (int i = 1; i <= n; i++) cin >> a[i];
  int len = 0;
  q[len] = -INF;
  for (int i = 1; i <= n; i++)
  {
      int l = 0, r = len;
      while (l < r)
      {
          int mid = l + r + 1 >> 1;
          if (q[mid] < a[i]) l = mid;
          else r = mid - 1;
      }
      len = max(r + 1, len);
      q[r + 1] = a[i];
  }
  cout << len;
}
\end{lstlisting}
\subsection{LCS}
\begin{lstlisting}
const int N = 1010;
int n, m, f[N][N];
char a[N], b[N];
int main()
{
  cin >> n >> m >> (a + 1) >> (b + 1);
  for (int i = 1; i <= n; i++)
      for (int j = 1; j <= m; j++)
      {
          f[i][j] = max(f[i - 1][j], f[i][j - 1]);
          if (a[i] == b[j])
              f[i][j] = max(f[i][j], f[i - 1][j - 1] + 1);
      }
  cout << f[n][m];
}
\end{lstlisting}
\section{Interval DP}
In this case we focus on an interval, whose sum of its elements can represent the answer we want to find:
\begin{lstlisting}
const int N = 310;
int n, s[N], f[N][N];
int main()
{
  cin >> n;
  for (int i = 1; i <= n; i++)
      cin >> s[i], s[i] += s[i - 1];
  for (int len = 2; len <= n; len++)
      for (int i = 1; i + len - 1 <= n; i++)
      {
          int l = i, r = i + len - 1;
          f[l][r] = INF;
          for (int k = l; k < r; k++)
              f[l][r] = min(f[l][r], f[l][k] + f[k + 1][r] + s[r] - s[l - 1]);
      }
  cout << f[1][n];
}
\end{lstlisting}
\section{Counting DP}
\begin{lstlisting}
const int N = 1010, M = 1e9 + 7;
int n, f[N][N];
int main()
{
  cin >> n;
  f[0][0] = 1;
  for (int i = 1; i <= n; i++)
      for (int j = 1; j <= i; j++)
          f[i][j] = (f[i - 1][j - 1] + f[i - j][j]) % M;
  int ans = 0;
  for (int i = 1; i <= n; i++)
      ans = (ans + f[n][i]) % M;
  cout << ans;
}
\end{lstlisting}
\section{Digit DP}
\begin{lstlisting}
// 求数 n 的位数
int get(int n)
{
  int res = 0;
  while (n) n /= 10, res++;
  return res;
}
int count(int n, int i)
{
  int res = 0, dgt = get(n);
  for (int j = 1; j <= dgt; j++)
  {
      // p 为当前遍历位次(第 j 位)的数大小 <10^(右边的数的位数)>, Ps: 从左往右(从高位到低位)
      // l 为第 j 位的左边的数,r 为右边的数,dj 为第 j 位上的数
      int p = pow(10, dgt - j), l = n / p / 10, r = n % p, dj = n / p % 10;
      // 求要选的数在 i 的左边的数小于 l 的情况:
      //      1)、当 i 不为 0 时 xxx : 0...0 ~ l - 1, 即 l * (右边的数的位数) == l * p 种选法
      //      2)、当 i 为 0 时 由于不能有前导零 故 xxx: 0....1 ~ l - 1, 即 (l - 1) * (右边的数的位数) == (l - 1) * p 种选法
      if (i) res += l * p;
      else res += (l - 1) * p;
      // 求要选的数在 i 的左边的数等于 l 的情况:(即视频中的xxx == l 时)
      //      1)、i > dj 时 0 种选法
      //      2)、i == dj 时 yyy : 0...0 ~ r 即 r + 1 种选法
      //      3)、i < dj 时 yyy : 0...0 ~ 9...9 即 10^(右边的数的位数) == p 种选法 */
      if (i == dj) res += r + 1;
      if (i < dj) res += p;
  }
  return res;
}
int main()
{
  int a, b;
  while (cin >> a >> b, a)
  {
      if (a > b) swap(a, b);
      for (int i = 0; i <= 9; ++i)
          cout << count(b, i) - count(a - 1, i) << ' ';
      // 利用前缀和思想:[l, r] 的和 = s[r] - s[l - 1]
      cout << '\n';
  }
}
\end{lstlisting}
\section{State Compression DP}
\begin{lstlisting}
const int N = 12, M = 1 << 12;
int n, m;
LL f[N][M];
bool st[M];
int main()
{
  while (cin >> n >> m, n || m)
  {
      memset(f, 0, sizeof f);
      for (int i = 0; i < 1 << n; i++)
      {
          st[i] = true;
          // 统计连续 0 的个数,若连续 0 为奇数个就不能正好放得下竖放的方格
          int cnt = 0;
          for (int j = 0; j < n && st[i]; j++)
              if (i >> j & 1)
              {
                  // 当前格子被使用
                  // 如果连续 0 的数量为奇数个,当前格子被使用的后果就是导致格子重合,所以不可取
                  if (cnt & 1)
                      st[i] = false;
                  // 刷新状态
                  cnt = 0;
              }
              else cnt++;
          // 最后再判断一次,防止漏判
          if (cnt & 1)
              st[i] = false;
      }
      // 没有摆放任何棋子的状态默认只有 1 种取法
      f[0][0] = 1;
      // 遍历每一列
      for (int i = 1; i <= m; i++)
          // 遍历当前列的每一种用二进制数字表示的摆放状态:1 指横向摆放,0 指空位
          for (int j = 0; j < 1 << n; j++)
              // 遍历上一列的每一种用二进制数字表示的摆放状态:1 指横向摆放,0 指空位
              for (int k = 0; k < 1 << n; k++)
                  // 满足两个条件:两列的摆放互不冲突;两列摆放状态的结合状态是一个可取的状态则累加情况数
                  if (!(j & k) && st[j | k])
                      f[i][j] += f[i - 1][k];
      // 输出摆放好第 m 列且第 (m + 1) 列没有任何方格的状态数
      cout << f[m][0] << '\n';
  }
}
\end{lstlisting}
\section{Tree DP}
\begin{lstlisting}
// Don't use I/O functions from stdio.h!!!
#define itn int
#define nit int
#define nti int
#define tin int
#define tni int
#define retrun return
#define reutrn return
#define rutren return
#define INF 0x3f3f3f3f
#include <bits/stdc++.h>
using namespace std;
typedef pair<int, int> PII;
typedef long long LL;

const int N = 6010;

int n;
int e[N], ne[N], happy[N], h[N], idx;
int f[N][2];
bool has_father[N];
void add(int a, int b)
{ e[idx] = b, ne[idx] = h[a], h[a] = idx++; }
void dfs(int u)
{
  f[u][1] = happy[u];
  for (int i = h[u]; ~i; i = ne[i])
  
      dfs(e[i]);
      f[u][0] += max(f[e[i]][0], f[e[i]][1]);
      f[u][1] += f[e[i]][0];
  }
}
int main()
{
  memset(h, -1, sizeof h);
  cin >> n;
  for (int i = 1; i <= n; i++) cin >> happy[i];
  for (int i = 0; i < n - 1; i++)
  {
      int a, b;
      cin >> a >> b;
      has_father[a] = true;
      add(b, a);
  }
  int root = 1;
  while (has_father[root]) root++;
  dfs(root);
  cout << max(f[root][0], f[root][1]);
}
\end{lstlisting}
\section{Memoized Search}
\begin{lstlisting}
const int N = 310;
int n, m, 
h[N][N], f[N][N], 
dx[4] = {0, 1, 0, -1}, dy[4] = {1, 0, -1, 0};
int dp(int x, int y)
{
  int &v = f[x][y];
  if (v != -1) return v;
  v = 1;
  for (int i = 0; i < 4; i++)
  {
      int a = x + dx[i], b = y + dy[i];
      if (a >= 1 && a <= n && b >= 1 && b <= m && h[a][b] < h[x][y])
          v = max(v, dp(a, b) + 1);
  }
  return v;
}
int main()
{
  cin >> n >> m;
  for (int i = 1; i <= n; i++)
      for (int j = 1; j <= m; j++)
          cin >> h[i][j];
  memset(f, -1, sizeof f);
  int res = 0;
  for (int i = 1; i <= n; i++)
      for (int j = 1; j <= m; j++)
          res = max(res, dp(i, j));
  cout << res;
}
\end{lstlisting}
