\chapter{Search \& Graph Theory}\label{chap:Graph}
\section{Representation of Tree \& Graph}
\subsection{Adjacency Matrix}
\begin{lstlisting}
// g[a][b] = a->b
\end{lstlisting}
\subsection{Adjacency List}
\begin{lstlisting}
int h[N], e[N], ne[N], idx;
void init() { memeset(h, -1, sizeof h); }
void add(int a, int b) { e[idx] = b, ne[idx] = h[a], h[a] = idx++ ; }
\end{lstlisting}
\section{DFS \& BFS}
\subsection{DFS}
\begin{lstlisting}
int dfs(int u)
{
    st[u] = true; // 表示点 u 已经被遍历过
    for (int i = h[u]; i != -1; i = ne[i])
    { int j = e[i]; if (!st[j]) dfs(j); }
}
\end{lstlisting}
\subsection{BFS}
\begin{lstlisting}
queue<int> q;
st[1] = true; q.push(1);
while (q.size())
{
    int t = q.front(); q.pop();
    for (int i = h[t]; i != -1; i = ne[i])
        if (!st[e[i]]) { st[e[i]] = true; q.push(e[i]); }
}
\end{lstlisting}
\section{Topological Sort}
\begin{lstlisting}
const int N = 100010;
int e[2 * N], ne[2 * N], h[N], d[N], idx;
int n, m, q[N];
void init() { memset(h, -1, sizeof h); }
void add(int a, int b) { e[idx] = b, ne[idx] = h[a], h[a] = idx++, d[b]++; }
bool topSort()
{
    int hh = 0, tt = -1;
    for (int i = 1; i <= n; i++)
        if (!d[i]) q[++tt] = i;
    while (hh <= tt)
        for (int i = h[q[hh++]]; ~i; i = ne[i])
            if (--d[e[i]] == 0) q[++tt] = e[i];
    return tt == n - 1;
}
\end{lstlisting}
\section{Shortest Path}
\subsection{Dijkstra}
\begin{lstlisting}
const int N = 1010;
int n, dist[N];
int h[N], w[N], e[N], ne[N], idx;
bool st[N];
void add(int a, int b, int c) { e[idx] = b, w[idx] = c, ne[idx] = h[a], h[a] = idx++; }
int dijkstra()      // 需要初始化 dist 与 h
{
    dist[1] = 0;
    priority_queue<PII, vector<PII>, greater<PII>> heap;
    heap.push({0, 1});
    while (heap.size())
    {
        auto t = heap.top();
        heap.pop();
        int ver = t.second, distance = t.first;
        if (st[ver]) continue;
        st[ver] = true;
        for (int i = h[ver]; i != -1; i = ne[i])
            if (dist[e[i]] > distance + w[i])
            {
                dist[e[i]] = distance + w[i];
                heap.push({dist[e[i]], e[i]});
            }
    }
    if (dist[n] == 0x3f3f3f3f) return -1;
    return dist[n];
}
\end{lstlisting}
\subsection{Bellman-Ford}
\begin{lstlisting}
const int N = 100010;
int n, m, dist[N], backup[N];
struct Edge
{
    int a, b, w;
}edges[N];
int bellman_ford()
{
    memset(dist, 0x3f, sizeof dist);
    dist[1] = 0;
    for (int i = 0; i < n; i ++ )
    {
        memcpy(backup, dist, sizeof dist);
        for (int j = 0; j < m; j++)
        {
            int a = edges[j].a, b = edges[j].b, w = edges[j].w;
            dist[b] = min(dist[b], backup[a] + w);
        }
    }
    if (dist[n] > 0x3f3f3f3f / 2) return -1;
    return dist[n];
}
\end{lstlisting}
\subsection{SPFA}
\begin{lstlisting}
const int N = 100010;
int n, m, dist[N];
int e[2 * N], ne[2 * N], w[2 * N], h[N], idx;
bool vis[N];
void spfa()     // 需要初始化 dist 与 h
{
    queue<int> q;
    q.push(1); vis[1] = true;
    while (q.size())
    {
        int t = q.front();
        q.pop();
        vis[t] = false;
        for (int i = h[t]; ~i; i = ne[i])
            if (dist[e[i]] > dist[t] + w[i])
            {
                dist[e[i]] = dist[t] + w[i];
                if (!vis[e[i]]) vis[e[i]] = true, q.push(j);
            }
    }
    dist[n] > INF / 2 ? cout << "impossible" : cout << dist[n];
}
\end{lstlisting}
\subsection{Detecting Negative Circle in SPFA}
\begin{lstlisting}
void spfa()     // 只需要初始化 h
{
    queue<int> q;
    // 基于虚拟原点假设,所有点放入队列
    for (int i = 1; i <= n; i++) q.push(i), st[i] = true;
    while (q.size())
    {
        int t = q.front();
        q.pop();
        vis[t] = false;
        for (int i = h[t]; ~i; i = ne[i])
            if (dist[e[i]] > dist[t] + w[i])
            {
                dist[e[i]] = dist[t] + w[i];
                // 新增
                cnt[j] = cnt[t] + 1;
                if (cnt[j] >= n) return true
                if (!st[j]) q.push(j),st[j] = true;
            }
    }
    return false;
}
\end{lstlisting}
\subsection{Floyd}
\begin{lstlisting}
const int N = 210;
int g[N][N], n, m, k;
int main()
{
    cin >> n >> m >> k;
    memset(g, 0x3f, sizeof g);
    for (int i = 1; i <= n; i++) g[i][i] = 0;
    while (m--)
    {
        int a, b, c;
        cin >> a >> b >> c;
        g[a][b] = min(g[a][b], c);
    }
    for (int k = 1; k <= n; k++)
        for (int i = 1; i <= n; i++)
            for (int j = 1; j <= n; j++)
                g[i][j] = min(g[i][k] + g[k][j], g[i][j]);
    // 后续代码略
    return 0;
}
\end{lstlisting}
\section{Minimum Spanning Tree}
\subsection{Prim}
\begin{lstlisting}
const int N = 510;
int n, m, g[N][N], dist[N];
bool vis[N];
void prim()
{
    int res = 0;
    for (int i = 0; i < n; i++)
    {
        int t = -1;
        for (int j = 1; j <= n; j++)
            if (!vis[j] && (t == -1 || dist[j] < dist[t])) t = j;
        if (i && dist[t] == INF) { res = INF; break; }
        if (i) res += dist[t];
        vis[t] = true;
        for (int j = 1; j <= n; j++) dist[j] = min(dist[j], g[t][j]);
    }
    res == INF ? cout << "impossible" : cout << res;
}
int main()
{
    memset(g, 0x3f, sizeof g);
    memset(dist, 0x3f, sizeof dist);
    cin >> n >> m;
    while (m--)
    {
        int a, b, c;
        cin >> a >> b >> c;
        g[a][b] = min(g[a][b], c);
        g[b][a] = min(g[b][a], c);
    }
    prim();
    return 0;
}
\end{lstlisting}
\subsection{Kruskal}
\begin{lstlisting}
const int N = 100010;
int n, m;
int p[N];
struct Edge
{
    int a, b, w;
    bool operator<(const Edge &e) const { return w < e.w; };
} edge[2 * N];
void init() { for (int i = 1; i <= n; i++) p[i] = i; }
int find(int x)
{
    if (x != p[x]) p[x] = find(p[x]);
    return p[x];
}
void merge(int x, int y) { p[find(x)] = find(y); }
void kruskal()
{
    int res = 0, cnt = 0;
    for (int i = 1; i <= m; i++)
        if (find(edge[i].a) != find(edge[i].b))
        {
            merge(edge[i].a, edge[i].b);
            res += edge[i].w;
            cnt++;
        }
    if (cnt < n - 1) res = INF;
    res == INF ? cout << "impossible" : cout << res;
}
int main()
{
    init();
    cin >> n >> m;
    for (int i = 1; i <= m; i++) cin >> edge[i].a >> edge[i].b >> edge[i].w;
    sort(edge + 1, edge + m + 1);
    kruskal();
    return 0;
}
\end{lstlisting}
\section{Bipartite Graph}
\subsection{Coloring Method}
To check if a given graph is bipartite.
\begin{lstlisting}
const int N = 100010, M = 200010;
int n, m;
int e[M], ne[M], h[N], color[N], idx;
bool dfs(int u, int c)
{
color[u] = c;
for (int i = h[u]; ~i; i = ne[i])
    if (color[e[i]] == -1)
    {
        if (!dfs(e[i], !c)) return false;
    }
    else if (color[e[i]] == c) return false;
return true;
}
bool check()
{
for (int i = 1; i <= n; i++)
    if (color[i] == -1)
        if (!dfs(i, 0)) return false;
return true;
}
int main()
{
// 注意另外初始化 h 与 color
cin >> n >> m;
while (m--)
{
    int a, b;
    cin >> a >> b;
    add(a, b), add(b, a);
}
// 其余过程略
}
\end{lstlisting}
\newpage
\subsection{Hungarian Algorithm}
To find the maximum matching for a given graph.
\begin{lstlisting}
const int N = 510, M = 100010;
int n1, n2, m;
int e[M], ne[M], h[N], match[N], idx;
bool vis[N];
bool find(int x)
{
for (int i = h[x]; ~i; i = ne[i])
    if (!vis[e[i]])
    {
        vis[e[i]] = true;
        if (match[e[i]] == 0 || find(match[e[i]]))
        {
            match[e[i]] = x;
            return true;
        }
    }
return false;
}
int main()
{
// 注意初始化 h
cin >> n1 >> n2 >> m;
while (m--)
{
    int a, b;
    cin >> a >> b;
    add(a, b);
}
int res = 0;
for (int i = 1; i <= n1; i++)
{
    memset(vis, false, sizeof vis);
    if (find(i)) res++;
}
cout << res;
return 0;
}
\end{lstlisting}
