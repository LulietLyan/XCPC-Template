\chapter{Advanced Data Structures}\label{chap:AdvancedDS}
\section{Binary Indexed Tree}
\begin{lstlisting}
// 支持区间修改、区间查询
// 利用变差分求二阶区间和
const int N = 100010;
int n, m, a[N];
LL tr1[N], tr2[N];
int lowbit(int x) { return x & -x; }
void add(LL tr[], LL x, LL c)
{
    for (int i = x; i <= n; i += lowbit(i))
        tr[i] += c;
}
LL sum(LL tr[], LL x)
{
    LL res = 0;
    for (int i = x; i; i -= lowbit(i))
        res += tr[i];
    return res;
}
LL prefix_sum(LL x)
{ return sum(tr1, x) * (x + 1) - sum(tr2, x); }
int main()
{
    cin >> n >> m;
    for (int i = 1; i <= n; i++)
        cin >> a[i];
    for (int i = 1; i <= n; i++)
    {
        int b = a[i] - a[i - 1];
        add(tr1, i, b);
        add(tr2, i, (LL)i * b);
    }
    while (m--)
    {
        char op[2];
        int l, r, d;
        cin >> op >> l >> r;
        if (*op == 'Q')
            cout << prefix_sum(r) - prefix_sum(l - 1) << '\n';
        else
        {
            cin >> d;
            add(tr1, l, d), add(tr2, l, (LL)l * d),
            add(tr1, r + 1, -d), 
            add(tr2, r + 1, (LL)-(r + 1) * d);
        }
    }
}
\end{lstlisting}
\section{Segment Tree}
\begin{lstlisting}
struct Node
{   // 可以维护任何满足区间加法的信息
    int l, r; LL sum, add; // 区间和/懒标记
} tr[N * 4];
void pushup(int u)  // 从上至下传递
{ tr[u].sum = tr[u << 1].sum + tr[u << 1 | 1].sum; }
void pushdown(int u)
{   // 从下至上传递
    auto &root = tr[u],
         &left = tr[u << 1],
         &right = tr[u << 1 | 1];
    if (root.add)
    {
        left.add += root.add,
            left.sum += (LL)(left.r - left.l + 1) * root.add;
        right.add += root.add,
            right.sum += (LL)(right.r - right.l + 1) * root.add;
        root.add = 0;
    }
}
void build(int u, int l, int r)
{   // 建树
    if (l == r) tr[u] = {l, r, w[r], 0};
    else
    {
        tr[u] = {l, r};
        int mid = l + r >> 1;
        build(u << 1, l, mid); // 左儿子
        build(u << 1 | 1, mid + 1, r); // 右儿子
        pushup(u); // 从下往上传递区间值
    }
}
void modify(int u, int l, int r, int d)
{   // 区间修改
    if (tr[u].l >= l && tr[u].r <= r)
    {
        tr[u].sum += (LL)(tr[u].r - tr[u].l + 1) * d;
        tr[u].add += d;
    }
    else
    {
        pushdown(u);
        int mid = tr[u].l + tr[u].r >> 1;
        if (l <= mid)
            modify(u << 1, l, r, d);
        if (r > mid)
            modify(u << 1 | 1, l, r, d);
        pushup(u);
    }
}
LL query(int u, int l, int r)
{   // 区间查询
    if (tr[u].l >= l && tr[u].r <= r)
        return tr[u].sum;
    pushdown(u);
    int mid = tr[u].l + tr[u].r >> 1;
    LL sum = 0;
    if (l <= mid)
        sum += query(u << 1, l, r);
    if (r > mid)
        sum += query(u << 1 | 1, l, r);
    return sum;
}
\end{lstlisting}