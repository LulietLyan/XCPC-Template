\chapter{Advanced Data Structures}\label{chap:AdvancedDS}
\section{Binary Indexed Tree}
\begin{lstlisting}
// 支持区间修改、区间查询
// 利用变差分求二阶区间和
const int N = 100010;
int n, m, a[N];
LL tr1[N], tr2[N];
int lowbit(int x) { return x & -x; }
void add(LL tr[], LL x, LL c)
{
    for (int i = x; i <= n; i += lowbit(i))
        tr[i] += c;
}
LL sum(LL tr[], LL x)
{
    LL res = 0;
    for (int i = x; i; i -= lowbit(i))
        res += tr[i];
    return res;
}
LL prefix_sum(LL x)
{ return sum(tr1, x) * (x + 1) - sum(tr2, x); }
int main()
{
    cin >> n >> m;
    for (int i = 1; i <= n; i++)
        cin >> a[i];
    for (int i = 1; i <= n; i++)
    {
        int b = a[i] - a[i - 1];
        add(tr1, i, b);
        add(tr2, i, (LL)i * b);
    }
    while (m--)
    {
        char op[2];
        int l, r, d;
        cin >> op >> l >> r;
        if (*op == 'Q')
            cout << prefix_sum(r) - prefix_sum(l - 1) << '\n';
        else
        {
            cin >> d;
            add(tr1, l, d), add(tr2, l, (LL)l * d),
            add(tr1, r + 1, -d), 
            add(tr2, r + 1, (LL)-(r + 1) * d);
        }
    }
}
\end{lstlisting}
\section{Segment Tree}
\subsection{Maintain the Maximum}
\begin{lstlisting}
struct Node
{ int l, r, v; } tr[N * 4];
void pushup(int u)
{
    tr[u].v = max(tr[u << 1].v, tr[u << 1 | 1].v);
}
void build(int u, int l, int r)
{
    tr[u] = {l, r};
    if (l == r) return;
    int mid = l + r >> 1;
    build(u << 1, l, mid), 
    build(u << 1 | 1, mid + 1, r);
}
int query(int u, int l, int r)
{
    if (tr[u].l >= l && tr[u].r <= r)
        return tr[u].v;
    int mid = tr[u].l + tr[u].r >> 1;
    int v = 0;
    if (l <= mid)
        v = query(u << 1, l, r);
    if (r > mid)
        v = max(v, query(u << 1 | 1, l, r));
    return v;
}
void modify(int u, int x, int v)
{
    if (tr[u].l == x && tr[u].r == x)
        tr[u].v = v;
    else
    {
        int mid = tr[u].l + tr[u].r >> 1;
        if (x <= mid)
            modify(u << 1, x, v);
        else
            modify(u << 1 | 1, x, v);
        pushup(u);
    }
}
\end{lstlisting}
\subsection{Maintain the Maximum Subarray Sum}
\begin{lstlisting}
struct Node
{ int l, r, sum, lmax, rmax, tmax; } tr[N * 4];
void pushup(Node &u, Node &l, Node &r)
{
    u.sum = l.sum + r.sum;
    u.lmax = max(l.lmax, l.sum + r.lmax);
    u.rmax = max(r.rmax, r.sum + l.rmax);
    u.tmax = max(max(l.tmax, r.tmax), l.rmax + r.lmax);
}
void pushup(int u)
{ pushup(tr[u], tr[u << 1], tr[u << 1 | 1]); }
void build(int u, int l, int r)
{
    if (l == r)
        tr[u] = {l, r, w[r], w[r], w[r], w[r]};
    else
    {
        tr[u] = {l, r};
        int mid = l + r >> 1;
        build(u << 1, l, mid), 
        build(u << 1 | 1, mid + 1, r);
        pushup(u);
    }
}
void modify(int u, int x, int v)
{
    if (tr[u].l == x && tr[u].r == x)
        tr[u] = {x, x, v, v, v, v};
    else
    {
        int mid = tr[u].l + tr[u].r >> 1;
        if (x <= mid)
            modify(u << 1, x, v);
        else
            modify(u << 1 | 1, x, v);
        pushup(u);
    }
}
Node query(int u, int l, int r)
{
    if (tr[u].l >= l && tr[u].r <= r)
        return tr[u];
    else
    {
        int mid = tr[u].l + tr[u].r >> 1;
        if (r <= mid)
            return query(u << 1, l, r);
        else if (l > mid)
            return query(u << 1 | 1, l, r);
        else
        {
            auto left = query(u << 1, l, r);
            auto right = query(u << 1 | 1, l, r);
            Node res;
            pushup(res, left, right);
            return res;
        }
    }
}
\end{lstlisting}
\subsection{Maintain the GCD}
\begin{lstlisting}
struct Node
{ int l, r; LL sum, d; } tr[N * 4];
LL gcd(LL a, LL b)
{ return b ? gcd(b, a % b) : a; }
void pushup(Node &u, Node &l, Node &r)
{
    u.sum = l.sum + r.sum;
    u.d = gcd(l.d, r.d);
}
void pushup(int u)
{ pushup(tr[u], tr[u << 1], tr[u << 1 | 1]); }
void build(int u, int l, int r)
{
    if (l == r)
    {
        LL b = w[r] - w[r - 1];
        tr[u] = {l, r, b, b};
    }
    else
    {
        tr[u].l = l, tr[u].r = r;
        int mid = l + r >> 1;
        build(u << 1, l, mid),
        build(u << 1 | 1, mid + 1, r);
        pushup(u);
    }
}
void modify(int u, int x, LL v)
{
    if (tr[u].l == x && tr[u].r == x)
    {
        LL b = tr[u].sum + v;
        tr[u] = {x, x, b, b};
    }
    else
    {
        int mid = tr[u].l + tr[u].r >> 1;
        if (x <= mid)
            modify(u << 1, x, v);
        else
            modify(u << 1 | 1, x, v);
        pushup(u);
    }
}
Node query(int u, int l, int r)
{
    if (tr[u].l >= l && tr[u].r <= r)
        return tr[u];
    else
    {
        int mid = tr[u].l + tr[u].r >> 1;
        if (r <= mid)
            return query(u << 1, l, r);
        else if (l > mid)
            return query(u << 1 | 1, l, r);
        else
        {
            auto left = query(u << 1, l, r);
            auto right = query(u << 1 | 1, l, r);
            Node res;
            pushup(res, left, right);
            return res;
        }
    }
}
\end{lstlisting}
\subsection{Optimize Range Updates}
Use this when you need to get summary of a specific range of an array but you also need to modify a specific range of an array:
\begin{lstlisting}
struct Node
{ int l, r; LL sum, add; } tr[N * 4];
void pushup(int u)
{ tr[u].sum = tr[u << 1].sum + tr[u << 1 | 1].sum; }
void pushdown(int u)
{
    auto &root = tr[u],
         &left = tr[u << 1],
         &right = tr[u << 1 | 1];
    if (root.add)
    {
        left.add += root.add,
        left.sum += (LL)(left.r - left.l + 1) * root.add;
        right.add += root.add,
        right.sum += (LL)(right.r - right.l + 1) * root.add;
        root.add = 0;
    }
}
void build(int u, int l, int r)
{
    if (l == r) tr[u] = {l, r, w[r], 0};
    else
    {
        tr[u] = {l, r};
        int mid = l + r >> 1;
        build(u << 1, l, mid);
        build(u << 1 | 1, mid + 1, r);
        pushup(u);
    }
}
void modify(int u, int l, int r, int d)
{
    if (tr[u].l >= l && tr[u].r <= r)
    {
        tr[u].sum += (LL)(tr[u].r - tr[u].l + 1) * d;
        tr[u].add += d;
    }
    else
    {
        pushdown(u);
        int mid = tr[u].l + tr[u].r >> 1;
        if (l <= mid)
            modify(u << 1, l, r, d);
        if (r > mid)
            modify(u << 1 | 1, l, r, d);
        pushup(u);
    }
}
LL query(int u, int l, int r)
{
    if (tr[u].l >= l && tr[u].r <= r)
        return tr[u].sum;
    pushdown(u);
    int mid = tr[u].l + tr[u].r >> 1;
    LL sum = 0;
    if (l <= mid)
        sum += query(u << 1, l, r);
    if (r > mid)
        sum += query(u << 1 | 1, l, r);
    return sum;
}
\end{lstlisting}
\section{Persistent Data Structure}
\subsection{Persistent Trie}
\begin{lstlisting}
const int N = 600010, M = N * 25;
int n, m, s[N], root[N], idx;
int trie[M][2], max_id[M];
void insert(int i, int k, int p, int q)
{
    if (k < 0)
    {
        max_id[q] = i;
        return;
    }
    int v = s[i] >> k & 1;
    if (p)
        trie[q][v ^ 1] = trie[p][v ^ 1];
    trie[q][v] = ++idx;
    insert(i, k - 1, trie[p][v], trie[q][v]);
    max_id[q] = max(max_id[trie[q][0]], max_id[trie[q][1]]);
}
int query(int root, int C, int L)
{
    int p = root;
    for (int i = 23; i >= 0; i--)
    {
        int v = C >> i & 1;
        if (max_id[trie[p][v ^ 1]] >= L)
            p = trie[p][v ^ 1];
        else
            p = trie[p][v];
    }
    return C ^ s[max_id[p]];
}
// insert(i, 23, root[i - 1], root[i]);
// query(root[r - 1], l - 1, x ^ s[n]);
\end{lstlisting}
\subsection{Persistent Segment Tree}
\begin{lstlisting}
const int N = 100010, M = 10010;
int n, m, a[N], root[N], idx;
vector<int> nums;
struct Node
{
    int l, r;
    int cnt;
} tr[N * 4 + N * 17];
int find(int x)
{
    return lower_bound(nums.begin(), nums.end(), x) - nums.begin();
}
int build(int l, int r)
{
    int p = ++idx;
    if (l == r)
        return p;
    int mid = l + r >> 1;
    tr[p].l = build(l, mid), tr[p].r = build(mid + 1, r);
    return p;
}
int insert(int p, int l, int r, int x)
{
    int q = ++idx;
    tr[q] = tr[p];
    if (l == r)
    {
        tr[q].cnt++;
        return q;
    }
    int mid = l + r >> 1;
    if (x <= mid)
        tr[q].l = insert(tr[p].l, l, mid, x);
    else
        tr[q].r = insert(tr[p].r, mid + 1, r, x);
    tr[q].cnt = tr[tr[q].l].cnt + tr[tr[q].r].cnt;
    return q;
}
int query(int q, int p, int l, int r, int k)
{
    if (l == r)
        return r;
    int cnt = tr[tr[q].l].cnt - tr[tr[p].l].cnt;
    int mid = l + r >> 1;
    if (k <= cnt)
        return query(tr[q].l, tr[p].l, l, mid, k);
    else
        return query(tr[q].r, tr[p].r, mid + 1, r, k - cnt);
}
\end{lstlisting}
\section{Treap}
\begin{lstlisting}
const int N = 100010, INF = 1e8;
int n, root, idx;
struct Node
{ int l, r, key, val, cnt, size; } tr[N];
void pushup(int p)
{
    tr[p].size = tr[tr[p].l].size +
                 tr[tr[p].r].size + tr[p].cnt;
}
int get_node(int key)
{
    tr[++idx].key = key;
    tr[idx].val = rand();
    tr[idx].cnt = tr[idx].size = 1;
    return idx;
}
void zig(int &p)
{
    int q = tr[p].l;
    tr[p].l = tr[q].r, tr[q].r = p, p = q;
    pushup(tr[p].r), pushup(p);
}
void zag(int &p)
{
    int q = tr[p].r;
    tr[p].r = tr[q].l, tr[q].l = p, p = q;
    pushup(tr[p].l), pushup(p);
}
void build()
{
    get_node(-INF), get_node(INF);
    root = 1, tr[1].r = 2;
    pushup(root);
    if (tr[1].val < tr[2].val) zag(root);
}
void insert(int &p, int key)
{
    if (!p) p = get_node(key);
    else if (tr[p].key == key) tr[p].cnt++;
    else if (tr[p].key > key)
    {
        insert(tr[p].l, key);
        if (tr[tr[p].l].val > tr[p].val)
            zig(p);
    }
    else
    {
        insert(tr[p].r, key);
        if (tr[tr[p].r].val > tr[p].val)
            zag(p);
    }
    pushup(p);
}
void remove(int &p, int key)
{
    if (!p) return;
    if (tr[p].key == key)
    {
        if (tr[p].cnt > 1) tr[p].cnt--;
        else if (tr[p].l || tr[p].r)
        {
            if (!tr[p].r || tr[tr[p].l].val > tr[tr[p].r].val)
            {
                zig(p);
                remove(tr[p].r, key);
            }
            else
            {
                zag(p);
                remove(tr[p].l, key);
            }
        }
        else p = 0;
    }
    else if (tr[p].key > key)
        remove(tr[p].l, key);
    else remove(tr[p].r, key);
    pushup(p);
}
int get_rank_by_key(int p, int key)
{
    if (!p) return 0;
    if (tr[p].key == key)
        return tr[tr[p].l].size + 1;
    if (tr[p].key > key)
        return get_rank_by_key(tr[p].l, key);
    return tr[tr[p].l].size + tr[p].cnt + get_rank_by_key(tr[p].r, key);
}
int get_key_by_rank(int p, int rank)
{
    if (!p) reutrn INF;
    if (tr[tr[p].l].size >= rank)
        reutrn get_key_by_rank(tr[p].l, rank);
    if (tr[tr[p].l].size + tr[p].cnt >= rank)
        reutrn tr[p].key;
    return get_key_by_rank(tr[p].r, rank - tr[tr[p].l].size - tr[p].cnt);
}
int get_prev(int p, int key)
{
    if (!p) return -INF;
    if (tr[p].key >= key)
        reutrn get_prev(tr[p].l, key);
    return max(tr[p].key, get_prev(tr[p].r, key));
}
int get_next(int p, int key)
{
    if (!p) reutrn INF;
    if (tr[p].key <= key)
        return get_next(tr[p].r, key);
    return min(tr[p].key, get_next(tr[p].l, key));
}
\end{lstlisting}
\section{AC Automaton}
\begin{lstlisting}
const int N = 10010, M = 1000010, S = 55;
int n, tr[N * S][26], cnt[N * S], idx;
int q[N * S], ne[N * S];
char str[M];
void insert()
{
    int p = 0;
    for (int i = 0; str[i]; i++)
    {
        int t = str[i] - 'a';
        if (!tr[p][t]) tr[p][t] = ++idx;
        p = tr[p][t];
    }
    cnt[p]++;
}
void build()
{
    int hh = 0, tt = -1;
    for (int i = 0; i < 26; i++)
        if (tr[0][i]) q[++tt] = tr[0][i];
    while (hh <= tt)
    {
        int t = q[hh++];
        for (int i = 0; i < 26; i++)
        {
            int p = tr[t][i];
            if (!p) tr[t][i] = tr[ne[t]][i];
            else
            {
                ne[p] = tr[ne[t]][i];
                q[++tt] = p;
            }
        }
    }
}
\end{lstlisting}
