\chapter{Advanced Math}\label{chap:AdvancedMath}
\section{Factorial Prime Factorization}
\begin{lstlisting}
const int N = 1000010;
int primes[N], cnt, n;
bool st[N];
void init(int n)
{
  for (int i = 2; i <= n; i++)
  {
    if (!st[i]) primes[cnt++] = i;
    for (int j = 0; primes[j] * i <= n; j++)
    {
      st[primes[j] * i] = true;
      if (i % primes[j] == 0) break;
    }
  }
}

int main()
{
  cin >> n;
  init(n);
  for (int i = 0; i < cnt; i++)
  {
    int p = primes[i];
    int s = 0;
    for (int j = n; j; j /= p) s += j / p;
    cout << p << ' ' << s << '\n';
  }
}
\end{lstlisting}
\section{Euler's Totient Function}
\subsection{GCD}
\begin{lstlisting}
const int N = 1e7 + 10;
int primes[N], cnt, phi[N];
bool st[N];
LL s[N];
void init(int n)
{
  for (int i = 2; i <= n; i++)
  {
      if (!st[i])
      {
          primes[cnt++] = i;
          phi[i] = i - 1;
      }
      for (int j = 0; primes[j] * i <= n; j++)
      {
          st[primes[j] * i] = true;
          if (i % primes[j] == 0)
          {
              phi[i * primes[j]] = phi[i] * primes[j];
              break;
          }
          phi[i * primes[j]] = phi[i] * (primes[j] - 1);
      }
  }
  for (int i = 1; i <= n; i++)
      s[i] = s[i - 1] + phi[i];
}
int main()
{
  int n; cin >> n;
  init(n);
  LL res = 0;
  for (int i = 0; i < cnt; i++)
  {
      int p = primes[i];
      res += s[n / p] * 2 + 1;
  }
}
\end{lstlisting}
\section{Matrix Multiplication}
\begin{lstlisting}
const int N = 3;
int n, m;
void mul(int c[], int a[], int b[][N])
{
  int temp[N] = {0};
  for (int i = 0; i < N; i++)
      for (int j = 0; j < N; j++)
          temp[i] = (temp[i] + (LL)a[j] * b[j][i]) % m;
  memcpy(c, temp, sizeof temp);
}
void mul(int c[][N], int a[][N], int b[][N])
{
  int temp[N][N] = {0};
  for (int i = 0; i < N; i++)
      for (int j = 0; j < N; j++)
          for (int k = 0; k < N; k++)
              temp[i][j] = (temp[i][j] + (LL)a[i][k] * b[k][j]) % m;
  memcpy(c, temp, sizeof temp);
}
int main()
{
  while (n)
  {
      if (n & 1) mul(f1, f1, a);
      mul(a, a, a); n >>= 1;
  }
}
\end{lstlisting}
