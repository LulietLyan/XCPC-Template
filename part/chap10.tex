\chapter{Advanced Math}\label{chap:AdvancedMath}
\section{Factorial Prime Factorization}
\begin{lstlisting}
const int N = 1000010;
int primes[N], cnt, n;
bool st[N];
void init(int n)
{
  for (int i = 2; i <= n; i++)
  {
    if (!st[i]) primes[cnt++] = i;
    for (int j = 0; primes[j] * i <= n; j++)
    {
      st[primes[j] * i] = true;
      if (i % primes[j] == 0) break;
    }
  }
}

int main()
{
  cin >> n;
  init(n);
  for (int i = 0; i < cnt; i++)
  {
    int p = primes[i];
    int s = 0;
    for (int j = n; j; j /= p) s += j / p;
    cout << p << ' ' << s << '\n';
  }
}
\end{lstlisting}
\section{Euler's Totient Function}
\subsection{GCD}
\begin{lstlisting}
const int N = 1e7 + 10;
int primes[N], cnt, phi[N];
bool st[N];
LL s[N];
void init(int n)
{
  for (int i = 2; i <= n; i++)
  {
      if (!st[i])
      {
          primes[cnt++] = i;
          phi[i] = i - 1;
      }
      for (int j = 0; primes[j] * i <= n; j++)
      {
          st[primes[j] * i] = true;
          if (i % primes[j] == 0)
          {
              phi[i * primes[j]] = phi[i] * primes[j];
              break;
          }
          phi[i * primes[j]] = phi[i] * (primes[j] - 1);
      }
  }
  for (int i = 1; i <= n; i++)
      s[i] = s[i - 1] + phi[i];
}
int main()
{
  int n; cin >> n;
  init(n);
  LL res = 0;
  for (int i = 0; i < cnt; i++)
  {
      int p = primes[i];
      res += s[n / p] * 2 + 1;
  }
}
\end{lstlisting}
\section{Matrix Multiplication}
\subsection{Simple Matrix Multiplication}
\begin{itemize}
  \item \textbf{Conclusion:} For the sequence $\{a_n\}$, if
\[
a_i = k_1 a_{i-1} + k_2 a_{i-2} + \cdots + k_i a_0,
\]
then
\[
\begin{bmatrix}
k_1 & k_2 & \cdots & k_{i-1} & k_i \\
1 & 0 & \cdots & 0 & 0 \\
0 & 1 & 0 & \cdots & 0 \\
\vdots & \ddots & \ddots & \ddots & \vdots \\
0 & \cdots & 0 & 1 & 0
\end{bmatrix}^n
\begin{bmatrix}
a_{i-1} \\
a_{i-2} \\
\vdots \\
a_1 \\
a_0
\end{bmatrix}
=
\begin{bmatrix}
a_{n+i-1} \\
a_{n+i-2} \\
\vdots \\
a_{n+1} \\
a_n
\end{bmatrix}
\]
\end{itemize}
\begin{lstlisting}
const int N = 3;
int n, m;
void mul(int c[], int a[], int b[][N])
{
  int temp[N] = {0};
  for (int i = 0; i < N; i++)
      for (int j = 0; j < N; j++)
          temp[i] = (temp[i] + (LL)a[j] * b[j][i]) % m;
  memcpy(c, temp, sizeof temp);
}
void mul(int c[][N], int a[][N], int b[][N])
{
  int temp[N][N] = {0};
  for (int i = 0; i < N; i++)
      for (int j = 0; j < N; j++)
          for (int k = 0; k < N; k++)
              temp[i][j] = (temp[i][j] + (LL)a[i][k] * b[k][j]) % m;
  memcpy(c, temp, sizeof temp);
}
int main()
{
  while (n)
  {
      if (n & 1) mul(f1, f1, a);
      mul(a, a, a); n >>= 1;
  }
}
\end{lstlisting}
\subsection{KMP \& Matrix Multiplication}
\begin{lstlisting}
const int N = 25;
int n, m, mod, ne[N], a[N][N];
char str[N];
void mul(int c[][N], int a[][N], int b[][N])
{
  int tmp[N][N];
  memset(tmp, 0, sizeof tmp);
  for (int i = 0; i < m; i++)
    for (int j = 0; j < m; j++)
      for (int k = 0; k < m; k++)
        tmp[i][j] = (tmp[i][j] + a[i][k] * b[k][j]) % mod;
  memcpy(c, tmp, sizeof tmp);
}
int qmi(int k)
{
  int f0[N][N] = {1};
  while (k)
  {
    if (k & 1) mul(f0, f0, a);
    mul(a, a, a);
    k >>= 1;
  }
  int res = 0;
  // 因为简化 f[0][j] 相当于 f[n][j]
  for (int j = 0; j < m; j++)
    res = (res + f0[0][j]) % mod;
  return res;
}
int main()
{
  cin >> n >> m >> mod;
  cin >> str + 1;
  for (int i = 2, j = 0; i <= m; i++)
  {
    while (j && str[j + 1] != str[i])
      j = ne[j];
    if (str[j + 1] == str[i]) j++;
    ne[i] = j;
  }
  for (int j = 0; j < m; j++)
    for (int c = '0'; c <= '9'; c++)
    {
      int k = j;
      while (k && str[k + 1] != c)
        k = ne[k];
      if (str[k + 1] == c) k++;
      if (k < m) a[j][k]++;
    }
  cout << qmi(n) << endl;
  return 0;
}
\end{lstlisting}
\section{Inclusion-Exclusion Principle}
\subsection{Mobius Function}
\begin{lstlisting}
const int N = 50010;
int primes[N], cnt, T, mobius[N], sum[N];
bool st[N];
void init(int n)
{
  mobius[1] = 1;
  for (int i = 2; i <= n; i++)
  {
    if (!st[i])
    {
      primes[cnt++] = i;
      mobius[i] = -1;
    }
    for (int j = 0; primes[j] <= n / i; j++)
    {
      int t = primes[j] * i;
      st[t] = true;
      if (i % primes[j] == 0)
      {
        mobius[t] = 0; break;
      }
      mobius[t] = mobius[i] * -1;
    }
  }
  for (int i = 1; i <= n; i++)
    sum[i] = sum[i - 1] + mobius[i];
}
int main()
{
  init(N - 1);
  cin >> T;
  while (T--)
  {
    int a, b, d;
    cin >> a >> b >> d;
    a /= d, b /= d;
    int n = min(a, b);
    LL res = 0;
    for (int l = 1, r; l <= n; l = r + 1)
    {
      r = min(n, min(a / (a / l), b / (b / l)));
      res += (sum[r] - sum[l - 1]) * (LL)(a / l) * (b / l);
    }
    cout << res << '\n';
  }
}
\end{lstlisting}
\section{Probability and Expected Value}
