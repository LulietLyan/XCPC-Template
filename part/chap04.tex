\chapter{Basic Math}\label{chap:BasicMath}
\section{Prime Numbers}
\subsection{Judging Prime Numbers}
$O(\sqrt{n})$
\begin{lstlisting}
bool is_prime(int x)
{
    if (x < 2) return false;
    for (int i = 2; i <= x / i; i ++ )
        if (x % i == 0) return false;
    return true;
}
\end{lstlisting}
\subsection{Prime Factorization}
\begin{lstlisting}
void divide(int x)
{
    for (int i = 2; i <= x / i; i ++ )
        if (x % i == 0)
        {   // 此条件成⽴时 i ⼀定是质数
            int s = 0;
            while (x % i == 0) x /= i, s ++ ;
            cout << i << ' ' << s << '\n';
        }
    if (x > 1) cout << x << ' ' << 1 << '\n'
}
\end{lstlisting}
\subsection{Euler's Sieve}
\begin{lstlisting}
int primes[N], cnt;
bool st[N]; 
void get_primes(int n)
{
    for (int i = 2; i <= n; i ++ )
    {
        if (!st[i]) primes[cnt++] = i;
        for (int j = 0; primes[j] <= n / i; j ++ )
        {
            st[primes[j] * i] = true;
            if (i % primes[j] == 0) break;
        }
    }
}
\end{lstlisting}
\section{Divisor}
\subsection{Find All Divisors}
\begin{lstlisting}
vector<int> get_divisors(int x)
{
    vector<int> res;
    for (int i = 1; i <= x / i; i ++ )
        if (x % i == 0)
        {
            res.push_back(i);
            if (i != x / i) res.push_back(x / i);
        }
    sort(res.begin(), res.end());
    return res;
}
\end{lstlisting}
\subsection{The Number of Divisors}
\begin{lstlisting}
const int mod = 1e9 + 7;
int n;
int main()
{
    cin >> n;
    unordered_map<int, int> h;
    while (n--)
    {
        int x;
        cin >> x;
        for (int i = 2; i <= x / i; i++)
            while (x % i == 0) { h[i]++; x = x / i; }
        if (x > 1) h[x]++;
    }
    long long res = 1;
    for (auto iter = h.begin(); iter != h.end(); iter++)
        res = res * (iter->second + 1) % mod;
    cout << res;
    return 0;
}
\end{lstlisting}
\subsection{The Sum of Divisors}
\begin{lstlisting}
const int mod = 1e9 + 7;
int n;
long long getSum(int x, int c)
{
    long long s = 1;
    while(c--) s = (s * x + 1) % mod;
    return s;
}
int main()
{
    cin >> n;
    unordered_map<int, int> h;
    while (n--)
    {
        int x;
        cin >> x;
        for (int i = 2; i <= x / i; i++)
            while (x % i == 0) { h[i]++; x = x / i; }
        if (x > 1) h[x]++;
    }
    long long res = 1;
    for (auto iter = h.begin(); iter != h.end(); iter++)
        res = res * getSum(iter->first, iter->second) % mod;
    cout << res;
    return 0;
}
\end{lstlisting}
\subsection{Euclidean Algorithm}
\begin{lstlisting}
int gcd(int a, int b) 
{ return a % b == 0 ? b : gcd(b, a % b); }
\end{lstlisting}
\section{Euler Function}
\subsection{Simple Method}
\begin{lstlisting}
int phi(int x)
{
    int res = x;
    for (int i = 2; i <= x / i; i ++ )
        if (x % i == 0)
        {
            res = res / i * (i - 1);
            while (x % i == 0) x /= i;
        }
    if (x > 1) res = res / x * (x - 1);
    return res;
}
\end{lstlisting}
\subsection{Euler's Sieve Method}
\begin{lstlisting}
const int N = 1000010;
int n, primes[N], phi[N], cnt;
bool st[N];
void getEuler()
{
    phi[1] = 1;
    for (int i = 2; i <= n; i++)
    {
        if (!st[i])
        {
            primes[cnt++] = i;
            // i 是质数,它只会被本身整除,所以直接赋值 i - 1
            phi[i] = i - 1;
        }
        for (int j = 0; primes[j] <= n / i; j++)
        {
            st[i * primes[j]] = true;
            if (i % primes[j] == 0)
            {
                // 如果 i % primes[j] == 0 成立表示 primes[j] 是 i 的最小质因子
                // 也是 primes[j] * i 的最小质因子
                // 1 - 1 / primes[j] 这一项在 phi[i] 中计算过了,只需将基数 N 修正为 primes[j] 倍
                phi[primes[j] * i] = phi[i] * primes[j];
                break;
            }
            // 否则,primes[j] 不是 i 的质因子,只是 primes[j] * i 的最小质因子
            // 不仅需要将基数 N 修正为 primes[j] 倍
            // 还需要补上 1 - 1 / primes[j] 的分子项,因此最终结果为 phi[i] * (primes[j] - 1)
            phi[primes[j] * i] = phi[i] * (primes[j] - 1);
        }
    }
}
\end{lstlisting}
\section{Exponentiating by Squaring}
\begin{lstlisting}
LL qmi(int m, int k, int p)
{
    LL res = 1 % p, t = m;
    while (k)
    {
        if (k&1) res = res * t % p;
        t = t * t % p;
        k >>= 1;
    }
    return res;
}
\end{lstlisting}
\section{Extended Euclidean Algorithm}
\begin{lstlisting}
int exgcd(int a, int b, int &x, int &y)
{
    if (!b)
    {
        x = 1;
        y = 0;
        return a;
    }
    int d = exgcd(b, a % b, y, x);
    y -= (a / b) * x;
    return d;
}
\end{lstlisting}
\section{Chinese Remainder Theorem}
\begin{lstlisting}
LL exgcd(LL a, LL b, LL &x, LL &y)
{
    if (!b) { x = 1, y = 0; return a; }
    LL d = exgcd(b, a % b, y, x);
    y -= a / b * x;
    return d;
}
int main()
{
    int n;
    cin >> n;
    LL x = 0, m1, a1;
    cin >> m1 >> a1;
    for (int i = 0; i < n - 1; i++)
    {
        LL m2, a2;
        cin >> m2 >> a2;
        LL k1, k2;
        LL d = exgcd(m1, m2, k1, k2);
        if ((a2 - a1) % d) { x = -1; break; }
        k1 *= (a2 - a1) / d;
        k1 = (k1 % (m2 / d) + m2 / d) % (m2 / d);
        x = k1 * m1 + a1;
        LL m = abs(m1 / d * m2);
        a1 = k1 * m1 + a1;
        m1 = m;
    }
    if (x != -1)
        x = (a1 % m1 + m1) % m1;
    cout << x << '\n';
    return 0;
}
\end{lstlisting}
\section{Gauss-Jordan Elimination}
\subsection{Linear Equation Group}
\begin{lstlisting}
int gauss()
{
    int c, r;
    for (c = 0, r = 0; c < n; c++)
    {
        int t = r;
        for (int i = r; i < n; i++)     // 找绝对值最大的行
            if (fabs(a[i][c]) > fabs(a[t][c]))
                t = i;
        if (fabs(a[t][c]) < eps)        // 此时没必要对该列该行处理
            continue;
        for (int i = c; i <= n; i++)
            swap(a[t][i], a[r][i]);     // 将绝对值最大的行换到最顶端
        for (int i = n; i >= c; i--)
            a[r][i] /= a[r][c];         // 将当前行的首位变成1
        for (int i = r + 1; i < n; i++) // 用当前行将下面所有的列消成0
            if (fabs(a[i][c]) > eps)
                for (int j = n; j >= c; j--)
                    a[i][j] -= a[r][j] * a[i][c];
        r++;
    }
    if (r < n)
    {
        for (int i = r; i < n; i++)
            if (fabs(a[i][n]) > eps)
                return 2; // 无解
        return 1;         // 有无穷多组解
    }
    for (int i = n - 1; i >= 0; i--)
        for (int j = i + 1; j < n; j++)
            a[i][n] -= a[i][j] * a[j][n];
    return 0;            // 有解
}
\end{lstlisting}
\subsection{XOR Linear Equation Group}
\begin{lstlisting}
int gauss()
{
    int c, r;
    for (c = 0, r = 0; c < n; c++)
    {
        int t = r;
        for (int i = r; i < n; i++)
            if (a[i][c])
                t = i;
        if (!a[t][c])
            continue;
        for (int i = c; i <= n; i++)
            swap(a[r][i], a[t][i]);
        for (int i = r + 1; i < n; i++)
            if (a[i][c])
                for (int j = n; j >= c; j--)
                    a[i][j] ^= a[r][j];
        r++;
    }
    if (r < n)
    {
        for (int i = r; i < n; i++)
            if (a[i][n])
                return 2;
        return 1;
    }
    for (int i = n - 1; i >= 0; i--)
        for (int j = i + 1; j < n; j++)
            a[i][n] ^= a[i][j] * a[j][n];
    return 0;
}
\end{lstlisting}
\section{Combinatorial Counting}
\subsection{Recurrence Relation}
\begin{lstlisting}
void init()
{
    for (int i = 0; i < N; i++)
        for (int j = 0; j <= i; j++)
            if (!j) c[i][j] = 1;
            else c[i][j] = (c[i - 1][j] + c[i - 1][j - 1]) % mod;
}
\end{lstlisting}
\subsection{Preprocessing \& Inverse Element}
\begin{lstlisting}
const int N = 100010, mod = 1e9 + 7;
int n, fact[N], infact[N];
int qmi(int a, int b, int p)
{
    int res = 1;
    while (b)
    {
        if (b & 1)
            res = (LL)res * a % p;
        a = (LL)a * a % p;
        b >>= 1;
    }
    return res;
}
int main()
{
    fact[0] = infact[0] = 1;
    for (int i = 1; i < N; i++)
    {
        fact[i] = (LL)fact[i - 1] * i % mod;
        infact[i] = (LL)infact[i - 1] * qmi(i, mod - 2, mod) % mod;
    }
    // 此后 C(a, b) = (LL)fact[a] * infact[b] % mod * infact[a - b] % mod
}
\end{lstlisting}
\subsection{Lucas Theorem}
\begin{lstlisting}
int qmi(int a, int k, int p)
{
    int res = 1 % p;
    while (k)
    {
        if (k & 1)
            res = (LL)res * a % p;
        a = (LL)a * a % p;
        k >>= 1;
    }
    return res;
}
int C(int a, int b, int p)
{
    if (a < b) return 0;
    LL x = 1, y = 1;
    // x = a * (a - 1) * (a - 2) * ... * (a - b + 1) = a! / (a - b)! (mod p)
    // y = 1 * 2 * ... * b = b! (mod p)
    for (int i = a, j = 1; j <= b; i--, j++)
    { x = (LL)x * i % p; y = (LL)y * j % p; }
    return x * (LL)qmi(y, p - 2, p) % p;
}
int lucas(LL a, LL b, int p)
{
    if (a < p && b < p)
        return C(a, b, p);
    return (LL)C(a % p, b % p, p) * lucas(a / p, b / p, p) % p;
}
\end{lstlisting}
\subsection{Factorization Method}
\begin{lstlisting}
const int N = 5010;
int n, primes[N], sum[N], cnt;
bool st[N];
void getPrimes(int n) { // 略 }
// 求 n! 中 p 的幂次
int get(int n, int p)
{
    int res = 0;
    while (n) { res += n / p; n /= p; }
    return res;
}
void mul(vector<int> &a, int b) { // 高精度乘,略 }
int main()
{
    int a, b;
    cin >> a >> b;
    getPrimes(a);
    for (int i = 0; i < cnt; i++)
    {
        int p = primes[i];
        sum[i] = get(a, p) - get(b, p) - get(a - b, p);
    }
    vector<int> res;
    res.push_back(1);
    for (int i = 0; i < cnt; i++)
        for (int j = 0; j < sum[i]; j++)
            mul(res, primes[i]);
    for (int i = res.size() - 1; i >= 0; i--)
        cout << res[i];
}
\end{lstlisting}
\subsection{Catalan Number}
\begin{lstlisting}
const int N = 100010, mod = 1e9 + 7;
int qmi(int a, int k, int p) { // 略 }
int main()
{
    int n;
    cin >> n;
    int a = n * 2, b = n, res = 1;
    for (int i = a; i > a - b; i--)
        res = (LL)res * i % mod;
    for (int i = 1; i <= b; i++)
        res = (LL)res * qmi(i, mod - 2, mod) % mod;
    res = (LL)res * qmi(n + 1, mod - 2, mod) % mod;
}
\end{lstlisting}
\section{Inclusion-Exclusion Principle}
% $$
% \mathbf{
% \begin{array}{l}
% \left|A_{1} \cup A_{2} \cup \cdots \cup A_{m}\right|= \\
% \sum_{1 \leq i \leq m}\left|A_{i}\right|-\sum_{1 \leq i<j \leq m}\left|A_{i} \cap A_{j}\right|+ \\
% \sum_{1 \leq i<j<k \leq m}\left|A_{i} \cap A_{j} \cap A_{k}\right|-\cdots+ \\
% (-1)^{m-1}\left|A_{1} \cap A_{2} \cap \cdots \cap A_{m}\right|
% \end{array}
% }
% $$
\begin{lstlisting}
const int N = 20;
int n, m, res = 0, p[N];
int main()
{
    cin >> n >> m;
    for (int i = 0; i < m; i++)
        cin >> p[i];
    // 使用二进制数字表示数字选取情况
    for (int i = 1; i < 1 << m; i++)
    {
        int t = 1, cnt = 0;
        // 遍历每个被选取的质数
        for (int j = 0; j < m; j++)
            if (i >> j & 1)
            {
                cnt++;
                // 一个质数能被选取的条件应该是其累乘积不超过目标数字
                if ((LL)t * p[j] > n)
                { t = -1; break; }
                t *= p[j];
            }
        if (t != -1)
            // 容斥原理公式中奇数个并集系数为 1,反之为 -1
            if (cnt % 2) res += n / t;
            else res -= n / t;
    }
    cout << res;
}
\end{lstlisting}
\section{Game Theory}
\subsection{NIM Game}
\begin{lstlisting}
const int N = 110, M = 100010;
int k, n,  s[N], f[M];
int sg(int x)
{
    if (f[x] != -1) return f[x];
    // 到达节点得 SG 函数集合
    unordered_set<int> S;
    // 能取走石子就说明能到达,并且递归向下求解
    for (int i = 0; i < k; i++)
    {
        int sum = s[i];
        if (x >= sum) S.insert(sg(x - sum));
    }
    // SG 从小到达遍历并返回,找到最小的、不包含在 SG 函数集合中的自然数
    for (int i = 0;; i++)
        if (!S.count(i))
            return f[x] = i;
}

int main()
{
    cin >> k;
    for (int i = 0; i < k; i++) cin >> s[i];
    cin >> n;
    memset(f, -1, sizeof f);
    int res = 0;
    // 每一堆石子都是一个入度为 0 的起始点
    for (int i = 0; i < n; i++)
    {
        int x;
        cin >> x;
        res ^= sg(x);
    }
    res ? cout << "Yes" : cout << "No";
    return 0;
}
\end{lstlisting}