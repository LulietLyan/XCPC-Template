\chapter{Advanced Graph Theory}\label{chap:AdvancedGraph}
\section{Detecting Negative Cycles}
\begin{lstlisting}
int n, m1, m2;
int h[N], e[M], w[M], ne[M], idx;
int dist[N], q[N], cnt[N];
bool st[N];
bool spfa()
{
    memset(dist, 0, sizeof dist);
    memset(cnt, 0, sizeof cnt);
    memset(st, 0, sizeof st);
    int hh = 0, tt = 0;
    for (int i = 1; i <= n; i++)
    { q[tt++] = i; st[i] = true; }
    while (hh != tt)
    {
        int t = q[hh++];
        if (hh == N) hh = 0;
        st[t] = false;
        for (int i = h[t]; ~i; i = ne[i])
        {
            int j = e[i];
            if (dist[j] > dist[t] + w[i])
            {
                dist[j] = dist[t] + w[i];
                cnt[j] = cnt[t] + 1;
                if (cnt[j] >= n)
                    return true;
                if (!st[j])
                {
                    q[tt++] = j;
                    if (tt == N) tt = 0;
                    st[j] = true;
                }
            }
        }
    }
    return false;
}
\end{lstlisting}
\section{SPFA-SLF}
Using deque to solve SPFA question.
\begin{lstlisting}
void spfa()
{
    memset(dist, 0x3f, sizeof dist);
    memset(st, 0, sizeof st);
    deque<int> q;
    q.push_back(s);
    st[s] = 1, dist[s] = 0;
    while (q.size())
    {
        int t = q.front();
        q.pop_front();
        st[t] = 0;
        for (int i = h[t]; ~i; i = ne[i])
        {
            int j = e[i];
            if (dist[j] > dist[t] + w[i])
            {
                dist[j] = dist[t] + w[i];
                if (!st[j])
                {
                    st[j] = true;
                    if (q.size() && dist[j] < dist[q.front()])
                        q.push_front(j);
                    else q.push_back(j);
                }
            }
        }
    }
}
\end{lstlisting}
\section{SPFA-Stack}
\begin{lstlisting}
bool spfa() 
{
    int hh = 0, tt = 1;
    memset(dist, -0x3f, sizeof dist);
    dist[0] = 0; q[0] = 0;
    while (hh != tt) 
    {
        int t = q[--tt];
        st[t] = false;
        for(int i = h[t]; ~i; i = ne[i])
        {
            int j = e[i];
            if (dist[j] < dist[t] + w[i]) 
            {
                dist[j] = dist[t] + w[i];
                cnt[j] = cnt[t] + 1;
                if (cnt[j] >= n + 1) return true;
                if (!st[j]) 
                {
                    st[j] = true; q[tt++] = j;
                }
            }
        }
    }
    return false;
}
\end{lstlisting}
\section{SPFA \& MIN \& MAX}
Using SPFA to maintain the minimum and maximum.
In this case we need \textbf{Original Graph} and \textbf{Reverse Graph}, in which we can use \textbf{type == 0} or \textbf{type == 1} to describe.
\begin{lstlisting}
void spfa(int h[], int dist[], int type)
{
    int hh = 0, tt = 1;
    if (type == 0)
    {
        memset(dist, 0x3f, sizeof dmin);
        dist[1] = w[1]; q[0] = 1;
    }
    else
    {
        memset(dist, -0x3f, sizeof dmax);
        dist[n] = w[n]; q[0] = n;
    }
    while (hh != tt)
    {
        int t = q[hh++];
        if (hh == N) hh = 0;
        st[t] = false;
        for (int i = h[t]; ~i; i = ne[i])
        {
            int j = e[i];
            if (type == 0 && dist[j] > min(dist[t], w[j]) || type == 1 && dist[j] < max(dist[t], w[j]))
            {
                if (type == 0)
                    dist[j] = min(dist[t], w[j]);
                else
                    dist[j] = max(dist[t], w[j]);
                if (!st[j])
                {
                    q[tt++] = j;
                    if (tt == N) tt = 0;
                    st[j] = true;
                }
            }
        }
    }
}
\end{lstlisting}
\section{Second Shortest Path}
\begin{lstlisting}
const int N = 1010, M = 20010;
struct Ver
{
    int id, type, dist;
    bool operator>(const Ver &W) const 
    { return dist > W.dist; }
};
int n, m, S, T, dist[N][2], cnt[N][2];
int h[N], e[M], w[M], ne[M], idx;
bool st[N][2];
void add(int a, int b, int c)
{
    e[idx] = b, w[idx] = c, ne[idx] = h[a], h[a] = idx++;
}
int dijkstra()
{
    memset(st, 0, sizeof st);
    memset(dist, 0x3f, sizeof dist);
    memset(cnt, 0, sizeof cnt);
    dist[S][0] = 0, cnt[S][0] = 1;
    priority_queue<Ver, vector<Ver>, greater<Ver>> heap;
    heap.push({S, 0, 0});
    while (heap.size())
    {
        Ver t = heap.top();
        heap.pop();
        int ver = t.id, type = t.type, distance = t.dist, count = cnt[ver][type];
        if (st[ver][type]) continue;
        st[ver][type] = true;
        for (int i = h[ver]; ~i; i = ne[i])
        {
            int j = e[i];
            if (dist[j][0] > distance + w[i])
            {
                dist[j][1] = dist[j][0], cnt[j][1] = cnt[j][0];
                heap.push({j, 1, dist[j][1]});
                dist[j][0] = distance + w[i], cnt[j][0] = count;
                heap.push({j, 0, dist[j][0]});
            }
            else if (dist[j][0] == distance + w[i])
                cnt[j][0] += count;
            else if (dist[j][1] > distance + w[i])
            {
                dist[j][1] = distance + w[i], cnt[j][1] = count;
                heap.push({j, 1, dist[j][1]});
            }
            else if (dist[j][1] == distance + w[i])
                cnt[j][1] += count;
        }
    }
    int res = cnt[T][0];
    if (dist[T][0] + 1 == dist[T][1])
        res += cnt[T][1];
    return res;
}
\end{lstlisting}
\section{Second Minimum Spanning Tree}
\begin{lstlisting}
const int N = 510, M = 10010;
int n, m, p[N], dist1[N][N], dist2[N][N];
struct Edge
{
    int a, b, w;
    bool f;
    bool operator<(const Edge &e) const 
    { return w < e.w; }
} edge[M];
int h[N], e[N * 2], w[N * 2], ne[N * 2], idx;
void add(int a, int b, int c)
{ e[idx] = b, w[idx] = c, ne[idx] = h[a], h[a] = idx++; }
int find(int x)
{
    if (p[x] != x) p[x] = find(p[x]);
    return p[x];
}
void dfs(int u, int fa, int maxd1, int maxd2, int d1[], int d2[])
{
    d1[u] = maxd1, d2[u] = maxd2;
    for (int i = h[u]; ~i; i = ne[i])
    {
        int j = e[i];
        if (j != fa)
        {
            int td1 = maxd1, td2 = maxd2;
            if (w[i] > td1)
                td2 = td1, td1 = w[i];
            else if (w[i] < td1 && w[i] > td2)
                td2 = w[i];
            dfs(j, u, td1, td2, d1, d2);
        }
    }
}
int main()
{
    cin >> n >> m;
    memset(h, -1, sizeof h);
    for (int i = 0; i < m; i++)
        cin >> edge[i].a >> edge[i].b >> edge[i].w;
    sort(edge, edge + m);
    for (int i = 1; i <= n; i++) p[i] = i;
    LL sum = 0;
    for (int i = 0; i < m; i++)
    {
        int a = edge[i].a, b = edge[i].b, w = edge[i].w;
        int pa = find(a), pb = find(b);
        if (pa != pb)
        {
            p[pa] = pb;
            sum += w;
            add(a, b, w), add(b, a, w);
            edge[i].f = true;
        }
    }
    for (int i = 1; i <= n; i++)
        dfs(i, -1, -1e9, -1e9, dist1[i], dist2[i]);
    LL res = 1e18;
    for (int i = 0; i < m; i++)
        if (!edge[i].f)
        {
            int a = edge[i].a, b = edge[i].b, w = edge[i].w;
            LL t;
            if (w > dist1[a][b])
                t = sum + w - dist1[a][b];
            else if (w > dist2[a][b])
                t = sum + w - dist2[a][b];
            res = min(res, t);
        }
}
\end{lstlisting}
\section{Difference Constraints}
\begin{itemize}
    \item \textbf{size == N}: Feasible Solution
    \item \textbf{size == 1}: Maximum/Minimum
    \item \textbf{Maximum}: Shortest Path
    \item \textbf{Minimum}: Longest Path
\end{itemize}
\subsection{Maximum-Shortest Path}
\begin{lstlisting}
bool spfa(int size)
{
    int hh = 0, tt = 0;
    memset(dist, 0x3f, sizeof dist);
    memset(st, 0, sizeof st);
    memset(cnt, 0, sizeof cnt);
    for (int i = 1; i <= size; i++)
    {
        q[tt++] = i;
        dist[i] = 0;
        st[i] = true;
    }
    while (hh != tt)
    {
        int t = q[hh++];
        if (hh == N) hh = 0;
        st[t] = false;
        for (int i = h[t]; ~i; i = ne[i])
        {
            int j = e[i];
            if (dist[j] > dist[t] + w[i])
            {
                dist[j] = dist[t] + w[i];
                cnt[j] = cnt[t] + 1;
                if (cnt[j] >= n)
                    return true;
                if (!st[j])
                {
                    st[j] = true;
                    q[tt++] = j;
                    if (tt == N) tt = 0;
                }
            }
        }
    }
    return false;
}
int main()
{
    // add(a, b, k) means x_b <= x_a + k
    // PROCESS
}
\end{lstlisting}
\subsection{Minimum-Longest Path}
\begin{lstlisting}
bool spfa(int size)
{
    int hh = 0, tt = 0;
    memset(dist, -0x3f, sizeof dist);
    memset(st, 0, sizeof st);
    memset(cnt, 0, sizeof cnt);
    for (int i = 1; i <= size; i++)
    {
        q[tt++] = i;
        dist[i] = 0;
        st[i] = true;
    }
    while (hh != tt)
    {
        int t = q[hh++];
        if (hh == N) hh = 0;
        st[t] = false;
        for (int i = h[t]; ~i; i = ne[i])
        {
            int j = e[i];
            if (dist[j] < dist[t] + w[i])
            {
                dist[j] = dist[t] + w[i];
                cnt[j] = cnt[t] + 1;
                if (cnt[j] >= n)
                    return false;
                if (!st[j])
                {
                    st[j] = true;
                    q[tt++] = j;
                    if (tt == N) tt = 0;
                }
            }
        }
    }
    return ture;
}
int main()
{
    // add(a, b, k) means x_a + k <= x_b
    // PROCESS
}
\end{lstlisting}
\section{LCA}
\begin{lstlisting}
int n, m, h[N], e[M], ne[M], idx;
int depth[N], fa[N][16], q[N];
void bfs(int root)
{
    memset(depth, 0x3f, sizeof depth);
    depth[0] = 0;
    depth[root] = 1;
    int hh = 0, tt = 0;
    q[0] = root;
    while (hh <= tt)
    {
        int t = q[hh++];
        for (int i = h[t]; ~i; i = ne[i])
        {
            int j = e[i];
            if (depth[j] > depth[t] + 1)
            {
                depth[j] = depth[t] + 1;
                q[++tt] = j;
                fa[j][0] = t;
                for (int k = 1; k <= 15; k++)
                    fa[j][k] = fa[fa[j][k - 1]][k - 1];
            }
        }
    }
}
int lca(int a, int b)
{
    if (depth[a] < depth[b])
        swap(a, b);
    for (int k = 15; k >= 0; k--)
        if (depth[fa[a][k]] >= depth[b])
            a = fa[a][k];
    if (a == b)
        return a;
    for (int k = 15; k >= 0; k--)
        if (fa[a][k] != fa[b][k])
        {
            a = fa[a][k];
            b = fa[b][k];
        }
    return fa[a][0];
}
\end{lstlisting}
\section{SCC}
\begin{lstlisting}
void tarjan(int u)
{
    dfn[u] = low[u] = ++timestap;
    stack[++top] = u, in_stk[u] = true;
    for(int i = h[u]; ~i; i = ne[i])
    {
        int j = e[i];
        if(!dfn[j])
        {
            tarjan(j);
            low[u] = min(low[u], low[j]);
        }
        else if(in_stk[j])
            low[u] = min(low[u], dfn[j]);
    }
    if(dfn[u] == low[u])
    {
        int y;
        ++scc_cnt;
        do
        {
            y = stk[top--];
            in_stk[y] = false;
            id[y] = scc_cnt;
        } while(y != u);
    }
} 
\end{lstlisting}
\section{DCC}
\subsection{e-DCC}
\begin{lstlisting}
void tarjan(int u, int from)
{
    dfn[u] = low[u] = ++timestamp;
    stk[++top] = u;
    for (int i = h[u]; ~i; i = ne[i])
    {
        int j = e[i];
        if (!dfn[j])
        {
            tarjan(j, i);
            low[u] = min(low[u], low[j]);
            if (dfn[u] < low[j])
                is_bridge[i] = is_bridge[i ^ 1] = true;
        }
        else if (i != (from ^ 1))
            low[u] = min(low[u], dfn[j]);
    }
    if (dfn[u] == low[u])
    {
        ++dcc_cnt;
        int y;
        do
        {
            y = stk[top--];
            id[y] = dcc_cnt;
        } while (y != u);
    }
}
\end{lstlisting}
\subsection{v-DCC}
\begin{lstlisting}
void tarjan(int u)
{
    dfn[u] = low[u] = ++timestamp;
    int cnt = 0;
    for (int i = h[u]; ~i; i = ne[i])
    {
        int j = e[i];
        if (!dfn[j])
        {
            tarjan(j);
            low[u] = min(low[u], low[j]);
            if (low[j] >= dfn[u])
                cnt++;
        }
        else
            low[u] = min(low[u], dfn[j]);
    }
    if (u != root) cnt++;
    ans = max(ans, cnt);
}
\end{lstlisting}
\subsection{Articulation Point}
\begin{lstlisting}
void tarjan(int u)
{
    dfn[u] = low[u] = ++timestamp;
    stk[++top] = u;
    if (u == root && h[u] == -1)
    {
        dcc_cnt++;
        dcc[dcc_cnt].push_back(u);
        return;
    }
    int cnt = 0;
    for (int i = h[u]; ~i; i = ne[i])
    {
        int j = e[i];
        if (!dfn[j])
        {
            tarjan(j);
            low[u] = min(low[u], low[j]);
            if (dfn[u] <= low[j])
            {
                cnt++;
                if (u != root || cnt > 1)
                    cut[u] = true;
                ++dcc_cnt;
                int y;
                do
                {
                    y = stk[top--];
                    dcc[dcc_cnt].push_back(y);
                } while (y != j);
                dcc[dcc_cnt].push_back(u);
            }
        }
        else
            low[u] = min(low[u], dfn[j]);
    }
}
\end{lstlisting}
\section{Bipartite Graph}
The maximum matching \\
(by the Hungarian algorithm) = \\
the minimum vertex cover = \\
total number of vertices - \\
maximum independent set = \\
total number of vertices - \\
minimum path cover.
\subsection{maximum matching}
\begin{lstlisting}
const int N = 110;
int n, m;
int dx[4] = {-1, 0, 1, 0}, dy[4] = {0, 1, 0, -1};
PII match[N][N];
bool g[N][N], st[N][N];
bool find(int x, int y)
{
    for (int i = 0; i < 4; i++)
    {
        int a = x + dx[i], b = y + dy[i];
        if (a && a <= n && b && b <= n && !g[a][b] && !st[a][b])
        {
            st[a][b] = true;
            PII t = match[a][b];
            if (t.x == -1 || find(t.x, t.y))
            {
                match[a][b] = {x, y};
                return true;
            }
        }
    }
    return false;
}
int main()
{
    // PROCESS
    memset(match, -1, sizeof match);
    int res = 0;
    for (int i = 1; i <= n; i++)
        for (int j = 1; j <= n; j++)
            if ((i + j) % 2 && !g[i][j])
            {
                memset(st, 0, sizeof st);
                if (find(i, j)) res++;
            }
    // PROCESS
}
\end{lstlisting}
\subsection{minimum vertex cover}
\begin{lstlisting}
const int N = 110;
int n, m, k, match[N];
bool g[N][N], st[N];
bool find(int x)
{
    for (int i = 0; i < m; i++)
        if (!st[i] && g[x][i])
        {
            st[i] = true;
            if (match[i] == -1 || find(match[i]))
            {
                match[i] = x;
                return true;
            }
        }
    return false;
}
int main()
{
    while (cin >> n, n)
    {
        cin >> m >> k;
        memset(g, 0, sizeof g);
        memset(match, -1, sizeof match);
        while (k--)
        {
            int t, a, b;
            cin >> t >> a >> b;
            if (!a || !b) continue;
            g[a][b] = true;
        }
        int res = 0;
        for (int i = 0; i < n; i++)
        {
            memset(st, 0, sizeof st);
            if (find(i)) res++;
        }
        cout << res << '\n';
    }
}
\end{lstlisting}
\subsection{maximum independent set}
\begin{lstlisting}
const int N = 110;
int n, m, k;
PII match[N][N];
bool g[N][N], st[N][N];
int dx[8] = {-2, -1, 1, 2, 2, 1, -1, -2};
int dy[8] = {1, 2, 2, 1, -1, -2, -2, -1};
bool find(int x, int y)
{
    for (int i = 0; i < 8; i++)
    {
        int a = x + dx[i], b = y + dy[i];
        if (a < 1 || a > n || b < 1 || b > m)
            continue;
        if (g[a][b]) continue;
        if (st[a][b]) continue;
        st[a][b] = true;
        PII t = match[a][b];
        if (t.x == 0 || find(t.x, t.y))
        {
            match[a][b] = {x, y};
            return true;
        }
    }
    return false;
}
int main()
{
    // PROCESS
    int res = 0;
    for (int i = 1; i <= n; i++)
        for (int j = 1; j <= m; j++)
        {
            if (g[i][j] || (i + j) % 2)
                continue;
            memset(st, 0, sizeof st);
            if (find(i, j)) res++;
        }
    cout << n * m - k - res << '\n';
}
\end{lstlisting}
\subsection{minimum path cover}
\begin{lstlisting}
const int N = 210, M = 30010;
int n, m, match[N];
bool d[N][N], st[N];
bool find(int x)
{
    for (int i = 1; i <= n; i++)
        if (d[x][i] && !st[i])
        {
            st[i] = true;
            int t = match[i];
            if (t == 0 || find(t))
            {
                match[i] = x;
                return true;
            }
        }
    return false;
}
int main()
{
    // 传递闭包
    for (int k = 1; k <= n; k++)
        for (int i = 1; i <= n; i++)
            for (int j = 1; j <= n; j++)
                d[i][j] |= d[i][k] & d[k][j];
    int res = 0;
    for (int i = 1; i <= n; i++)
    {
        memset(st, 0, sizeof st);
        if (find(i)) res++;
    }
    cout << n - res;
}
\end{lstlisting}
\section{Eulerian Circuit \& Eulerian Path}
\subsection{Eulerian Circuit}
\begin{lstlisting}
int type, n, m;
int h[N], e[M], ne[M], idx;
bool used[M];
int ans[M], cn, din[N], dout[N];
void add(int a, int b)
{
    e[idx] = b, ne[idx] = h[a], h[a] = idx++;
}
void dfs(int u)
{
    for (int &i = h[u]; ~i;)
    {
        if (used[i])
        { i = ne[i]; continue; }
        used[i] = true;
        if (type == 1) used[i ^ 1] = true;
        int t;
        if (type == 1)
        {
            t = i / 2 + 1;
            if (i & 1) t = -t;
        }
        else t = i + 1;
        int j = e[i];
        i = ne[i];
        dfs(j);
        ans[++cnt] = t;
    }
}
int main()
{
    cin >> type >> n >> m;
    memset(h, -1, sizeof h);
    for (int i = 0; i < m; i++)
    {
        int a, b;
        cin >> a >> b;
        add(a, b);
        if (type == 1) add(b, a);
        din[b]++, dout[a]++;
    }
    if (type == 1)
    {
        for (int i = 1; i <= n; i++)
            if (din[i] + dout[i] & 1)
            {
                cout << "NO\n";
                return 0;
            }
    }
    else
    {
        for (int i = 1; i <= n; i++)
            if (din[i] != dout[i])
            {
                cout << "NO\n";
                return 0;
            }
    }
    for (int i = 1; i <= n; i++)
        if (h[i] != -1) { dfs(i); break; }
}
\end{lstlisting}
\subsection{Eulerian Path}
\begin{lstlisting}
const int N = 510;
int n = 500, m, g[N][N];
int ans[1100], cnt, d[N];
void dfs(int u)
{
    for (int i = 1; i <= n; i++)
        if (g[u][i])
        {
            g[u][i]--, g[i][u]--;
            dfs(i);
        }
    ans[++cnt] = u;
}
int main()
{
    cin >> m;
    while (m--)
    {
        int a, b;
        cin >> a >> b;
        g[a][b]++, g[b][a]++;
        d[a]++, d[b]++;
    }
    int start = 1;
    while (!d[start]) ++start;
    for (int i = 1; i <= 500; i++)
        if (d[i] % 2)
        { start = i; break; }
    dfs(start);
}
\end{lstlisting}