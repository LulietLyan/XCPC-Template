\chapter{Basic Algorithm}\label{chap:BasicAlgorithm}

\section{Quick Sort}

Sort the given array from index 1 to n.

\begin{lstlisting}
void quick_sort(int l, int r)
{
    if (l >= r) return;
    int x = a[(l + r) >> 1], i = l - 1, j = r + 1;
    while (i < j)
    {
        do i++; while (a[i] < x);
        do j--; while (a[j] > x);
        if (i < j) swap(a[i], a[j]);
    }
    quick_sort(l, j);
    quick_sort(j + 1, r);
    return;
}
\end{lstlisting}
\section{Binary Search}
\begin{lstlisting}
// 区间 [l, r] 被划分成 [l, mid] 和 [mid + 1, r] 时使用
// 大于等于区间的最小值,check 应为 target <= a[mid]
int bsearch_1(int l, int r)
{
    while (l < r)
    {
        int mid = l + r >> 1;
        if (check(mid)) r = mid;
        else l = mid + 1;
    }
    return l;
}
// 区间 [l, r] 被划分成 [l, mid - 1] 和 [mid, r] 时使用
// 小于等于区间的最大值,check 应为 target >= a[mid]
int bsearch_2(int l, int r)
{
    while (l < r)
    {
        // 为什么要 l + r + 1:因为 l 的更新条件是 mid 本身
        // 当 r == l + 1 时 mid 向下取整必定取 l,有可能在满足 check(mid) 时导致无限循环
        int mid = l + r + 1 >> 1;
        if (check(mid)) l = mid;
        else r = mid - 1;
    }
    return l;
}
// 浮点数二分
double bsearch_3(double l, double r)
{
    // eps 表示精度,取决于题目对精度的要求
    const double eps = 1e-6;
    while (r - l > eps)
    {
        double mid = (l + r) / 2;
        if (check(mid))  r = mid;
        else l = mid;
    }
    return l;
}
\end{lstlisting}
\section{Ternary Search}
\begin{lstlisting}
// 整数三分
void tsearch_1(int l, int r)
{
    while (l < r)
    {
        int lmid = l + (r - l) / 3, rmid = r - (r - l) / 3;
        lans = cal(lmid), rans = cal(rmid);
        if (lans <= rans) r = rmid - 1;
        else l = lmid + 1;
        if (lans <= rans) l = lmid + 1;
        else r = rmid - 1;
    }
    // 求凹函数的极小值
    cout << min(lans, rans) << endl;
    // 求凸函数的极大值
    cout << max(lans, rans) << endl;
}
// 浮点数三分
void tsearch_2(int l, int r)
{
    const double eps = 1e-6;
    while (r - l < eps)
    {
        double lmid = l + (r - l) / 3;
        double rmid = r - (r - l) / 3;
        lans = cal(lmid), rans = cal(rmid);
        // 求凹函数的极小值
        if (lans <= rans) r = rmid;
        else l = lmid;
        // 求凸函数的极大值
        if (lans <= rans) l = lmid;
        else r = rmid;
    }
}
\end{lstlisting}
\section{High Precision}
\subsection{High Precision Add}
\begin{lstlisting}
string s1, s2;
vector<int> a, b, c;
void add(vector<int> &a, vector<int> &b)
{
    if (a.size() < b.size())
    { add(b, a); return; }
    int t = 0;
    for (int i = 0; i < a.size(); i++)
    {
        t += a[i];
        if (i < b.size()) t += b[i];
        c.push_back(t % 10);
        t /= 10;
    }
    while (t)
        c.push_back(t % 10), t /= 10;
}
int main()
{
    cin >> s1 >> s2;
    for (int i = s1.size() - 1; i >= 0; i--)
        a.push_back(s1[i] - '0');
    for (int i = s2.size() - 1; i >= 0; i--)
        b.push_back(s2[i] - '0');
    add(a, b);
    for (int i = c.size() - 1; i >= 0; i--)
        cout << c[i];
    return 0;
}
\end{lstlisting}
\subsection{High Precision Subsection}
\begin{lstlisting}
vector<int> a, b, c;
string s1, s2;
void sub(vector<int> &a, vector<int> &b)
{
    int t = 0;
    for (int i = 0; i < a.size(); i++)
    {
        t = a[i] - t;
        if (i < b.size()) t -= b[i];
        c.push_back((t + 10) % 10);
        if (t < 0) t = 1;
        else t = 0;
    }
    while (c.size() > 1 && c.back() == 0)
        c.pop_back();
}
int main()
{
    cin >> s1 >> s2;
    for (int i = s1.size() - 1; i >= 0; i--)
        a.push_back(s1[i] - '0');
    for (int i = s2.size() - 1; i >= 0; i--)
        b.push_back(s2[i] - '0');
    if (s1.size() < s2.size())
        cout << '-', sub(b, a);
    else if (s1.size() == s2.size() && s1 < s2)
        cout << '-', sub(b, a);
    else sub(a, b);
    for (int i = c.size() - 1; i >= 0; i--)
        cout << c[i];
    return 0;
}
\end{lstlisting}
\subsection{High Precision Multiply}
\begin{lstlisting}
string s1, s2;
vector<int> a, c;
int b;
void mul(vector<int> &a, int b)
{
    for (int i = 0, t = 0; i < a.size() || t; i++)
    {
        if (i < a.size()) t += a[i] * b;
        c.push_back(t % 10);
        t /= 10;
    }
    while (c.size() > 1 && c.back() == 0)
        c.pop_back();
}
int main()
{
    cin >> s1 >> b;
    for (int i = s1.size() - 1; i >= 0; i--)
        a.push_back(s1[i] - '0');
    mul(a, b);
    for (int i = c.size() - 1; i >= 0; i--)
        cout << c[i];
    return 0;
}
\end{lstlisting}
\subsection{High Precision Divide}
\begin{lstlisting}
string s1, s2;
vector<int> a, c;
int b, r;
void divide(vector<int> &a, int b, int &r)
{
    r = 0;
    for (int i = a.size() - 1; i >= 0; i--)
    {
        r = r * 10 + a[i];
        c.push_back(r / b);
        r %= b;
    }
    reverse(c.begin(), c.end());
    while (c.size() > 1 && c.back() == 0)
        c.pop_back();
}
int main()
{
    cin >> s1 >> b;
    for (int i = s1.size() - 1; i >= 0; i--)
        a.push_back(s1[i] - '0');
    divide(a, b, r);
    for (int i = c.size() - 1; i >= 0; i--)
        cout << c[i];
    cout << '\n' << r;
    return 0;
}
\end{lstlisting}
\section{Prefix Sum \& Difference Array}
\subsection{1D Prefix Sum}
\begin{lstlisting}
S[i] = a[1] + a[2] + ... a[i]
a[l] + ... + a[r] = S[r] - S[l - 1]
\end{lstlisting}
\subsection{2D Prefix Sum}
\begin{lstlisting}
// S[i, j] = i 行 j 列左上部分所有元素和为:
s[i - 1][j] + s[i][j - 1] - s[i - 1][j - 1] + a[i][j]
// 以 (x1, y1) 为左上角,(x2, y2) 为右下角的子矩阵的和为:
S[x2][y2] - S[x1 - 1][y2] - S[x2][y1 - 1] + S[x1 - 1][y1 - 1]
\end{lstlisting}
\subsection{1D Difference Array}
\begin{lstlisting}
const int N = 100010;
int n, m;
int a[N], b[N];
void insert(int l, int r, int c)
{ b[l] += c; b[r + 1] -= c; }
int main()
{
    cin >> n >> m;
    for (int i = 1; i <= n; i++)
        cin >> a[i];
    for (int i = 1; i <= n; i++)
        insert(i, i, a[i]);
    while (m--)
    {
        int l, r, c;
        cin >> l >> r >> c;
        insert(l, r, c);
    }
    for (int i = 1; i <= n; i++)
        b[i] += b[i - 1], 
        cout << b[i] << ' ';
    return 0;
}
\end{lstlisting}
\subsection{2D Difference Array}
\begin{lstlisting}
const int N = 1010;
int n, m, q, a[N][N], b[N][N];
void insert(int x1, int y1, int x2, int y2, int c)
{
    b[x1][y1] += c;
    b[x2 + 1][y2 + 1] += c;
    b[x1][y2 + 1] -= c;
    b[x2 + 1][y1] -= c;
}
int main()
{
    cin >> n >> m >> q;
    for (int i = 1; i <= n; i++)
        for (int j = 1; j <= m; j++)
            cin >> a[i][j];
    for (int i = 1; i <= n; i++)
        for (int j = 1; j <= m; j++)
            insert(i, j, i, j, a[i][j]);
    while (q--)
    {
        int x1, x2, y1, y2, c;
        cin >> x1 >> y1 >> x2 >> y2 >> c;
        insert(x1, y1, x2, y2, c);
    }
    // 其他过程略
}
\end{lstlisting}