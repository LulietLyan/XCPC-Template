\chapter{Basic Data Structures}\label{chap:BasicDaraStructures}
\section{Linked List}
\subsection{Singly Linked List}
\begin{lstlisting}
const int N = 100010;
int n, h[N], e[N], ne[N], idx = 1;
void init() { ne[0] = -1; }
void insert(int k, int x) //  第 k 个节点后插入
{ e[idx] = x, ne[idx] = ne[k], ne[k] = idx++; }
void del(int k) // 第 k 个节点后删除
{ ne[k] = ne[ne[k]]; }
\end{lstlisting}
\subsection{Bidirectional Linked List}
\begin{lstlisting}
const int N = 100010;
int n, r[N], l[N], e[N], idx = 2;
void init() { r[0] = 1; l[1] = 0; }
void insert(int k, int x) // 第 k 个节点后插入
{
  e[idx] = x;
  r[idx] = r[k];
  l[idx] = k;
  l[r[k]] = idx;
  r[k] = idx++;
}
void remove(int k) // 删除 k 本身
{ r[l[k]] = r[k]; l[r[k]] = l[k]; }
\end{lstlisting}
\section{Stack \& Queue}
\subsection{Monotonic Stack}
\begin{lstlisting}
// 常见模型:找出每个数左边离它最近的比它大/小的数
int tt = 0;
for (int i = 1; i <= n; i ++ )
{
  while (tt && check(stk[tt], i)) tt -- ;
  stk[++tt] = i;
}
\end{lstlisting}
\subsection{Monotonic Queue}
\begin{lstlisting}
// 常见模型:找出滑动窗口中的最大值/最小值
int hh = 0, tt = -1;
for (int i = 0; i < n; i ++ )
{
  while (hh <= tt && check_out(q[hh]))
      hh++; // 判断队头是否滑出窗口
  while (hh <= tt && check(q[tt], i))
      tt-- ;
  q[++tt] = i;
}
\end{lstlisting}
\section{KMP}
\begin{lstlisting}
const int N = 100010, M = 1000010;
int n, m;
char p[N], s[M];
void getNext(int ne[])
{
  for (int i = 2, j = 0; i <= n; i++)
  {
      while (j && p[j + 1] != p[i])
          j = ne[j];
      if (p[j + 1] == p[i]) j++;
      ne[i] = j;
  }
}
int KMP()
{
  int *ne = new int[n + 1];
  getNext(ne);
  for (int i = 1, j = 0; i <= m; i++)
  {
      while (j && p[j + 1] != s[i])
          j = ne[j];
      if (p[j + 1] == s[i]) j++;
      if (j == n) cout << i - n << ' ';
  }
  return -1;
}
\end{lstlisting}

\section{Trie}
\begin{lstlisting}
const int N = 100010;
int trie[N][26], cnt[N], idx = 0;
void insert(string &str)    // 插入到 Trie 数组
{
  int p = 0;
  for (auto c : str)
  {
      int u = c - 'a';
      if (!trie[p][u])
          trie[p][u] = ++idx;
      p = trie[p][u];
  }
  cnt[p]++;
}
int query(string &str)      // 查询字符串出现的次数
{
  int p = 0;
  for (auto c : str)
  {
      int u = c - 'a';
      if (!trie[p][u]) return 0;
      p = trie[p][u];
  }
  return cnt[p];
}
\end{lstlisting}
\section{Disjoint-Set}
\begin{lstlisting}
const int N = 100010;
int n, m, p[N], Size[N], D[N];
void init()
{
  for (int i = 1; i <= n; i ++ )
      p[i] = i, Size[i] = 1, D[i] = 0;
}
int find(int x)
{
  if (p[x] != x)
  {
      int u = find(p[x]);
      D[x] += D[p[x]];  // 视具体情况计算
      p[x] = u;
  }
  return p[x];
}
void merge(int a, int b, int distance)
{
  int x = find(a), y = find(b);
  if(x != y)
  {
      p[x] = y; 
      D[x] = distance;  // 视具体情况计算
      Size[y] += Size[x];
  }
}
\end{lstlisting}
\section{Hash}
\subsection{Simple Hash}
\begin{lstlisting}
// (1) 拉链法
int h[N], e[N], ne[N], idx;
void insert(int x)
{
  int k = (x % N + N) % N;
  e[idx] = x, ne[idx] = h[k], h[k] = idx ++ ;
}
bool find(int x)
{
  for (int i = h[(x % N + N) % N]; i != -1; i = ne[i])
      if (e[i] == x) return true;
  return false;
}
// (2) 开放寻址法
int find(int x)
{
  int t = (x % N + N) % N;
  while (h[t] != null && h[t] != x)
  { t ++ ; if (t == N) t = 0; }
  return t;
}
\end{lstlisting}
\subsection{String Hash}
\begin{lstlisting}
typedef unsigned long long ULL;
ULL h[N], p[N];
void init()
{
  p[0] = 1;
  for (int i = 1; i <= n; i ++ ) { h[i] = h[i - 1] * P + str[i]; p[i] = p[i - 1] * P; }
}
ULL get(int l, int r) { return h[r] - h[l - 1] * p[r - l + 1]; }
\end{lstlisting}
\section{STL}
\begin{lstlisting}
// vector
size()      返回元素个数
empty()     返回是否为空
clear()     清空
front()/back()
push_back()/pop_back()
begin()/end()
[]
支持比较运算,按字典序
// pair<int, int>
first       第一个元素
second      第二个元素
支持比较运算,以first为第一关键字,以second为第二关键字(字典序)
// string
size()/length()  返回字符串长度
empty()
clear()
substr(起始下标,(子串长度))  返回子串
c_str()  返回字符串所在字符数组的起始地址
// queue
size()
empty()
push()      向队尾插入一个元素
front()     返回队头元素
back()      返回队尾元素
pop()       弹出队头元素
// priority_queue
size()
empty()
push()      插入一个元素
top()       返回堆顶元素
pop()       弹出堆顶元素
定义成小根堆的方式:priority_queue<int, vector<int>, greater<int>> q;
// stack
size()
empty()
push()      向栈顶插入一个元素
top()       返回栈顶元素
pop()       弹出栈顶元素
// deque
size()
empty()
clear()
front()/back()
push_back()/pop_back()
push_front()/pop_front()
begin()/end()
[]
// set, map, multiset, multimap:基于平衡二叉树(红黑树)动态维护有序序列
size()
empty()
clear()
begin()/end()
++, -- 返回前驱和后继,时间复杂度 O(logn)
// set/multiset
  insert()  插入一个数
  find()    查找一个数
  count()   返回某一个数的个数
  erase()
      (1) 输入是一个数x,删除所有x,O(k + logn)
      (2) 输入一个迭代器,删除这个迭代器
  lower_bound()/upper_bound()
      lower_bound(x)  返回大于等于x的最小的数的迭代器
      upper_bound(x)  返回大于x的最小的数的迭代器
// map/multimap
  insert()  插入的数是一个pair
  erase()   输入的参数是pair或者迭代器
  find()
  []        注意multimap不支持此操作。 时间复杂度是 O(logn)
  lower_bound()/upper_bound()
// unordered_set, unordered_map, unordered_multiset, unordered_multimap
增删改查的时间复杂度是 O(1)
不支持 lower_bound()/upper_bound(), 迭代器的++,--
// bitset
bitset<10000> s;
~, &, |, ^
>>, <<
==, !=
[]
count()     返回有多少个1
any()       判断是否至少有一个1
none()      判断是否全为0
set()       把所有位置成1
set(k, v)   将第k位变成v
reset()     把所有位变成0
flip()      等价于~
flip(k)     把第k位取反
\end{lstlisting}
