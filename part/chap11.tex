\chapter{Advanced DP}\label{chap:AdvancedDP}
\section{Advanced Linear DP}
\subsection{Two-pass grid collection problem}
In this case we run DP on two different roads at the same time:
\begin{lstlisting}
const int N = 15;
int n, w[N][N], f[N * 2][N][N];
int main()
{
  cin >> n;
  // INPUT w[N][N]
  for (int k = 2; k <= n * 2; k++)
  for (int i1 = 1; i1 <= k; i1++)
    for (int i2 = 1; i2 <= k; i2++)
    {
      int j1 = k - i1, j2 = k - i2;
      int t = w[i1][j1];
      if (i1 != i2) t += w[i2][j2];
      int &x = f[k][i1][i2];
      x = max(x, f[k - 1][i1 - 1][i2 - 1] + t);
      x = max(x, f[k - 1][i1 - 1][i2] + t);
      x = max(x, f[k - 1][i1][i2 - 1] + t);
      x = max(x, f[k - 1][i1][i2] + t);
    }
  cout << f[n * 2][n][n] << '\n';
  return 0;
}
\end{lstlisting}
\section{Advanced LIS}
\subsection{Longest Bitonic Subsequence}
\begin{lstlisting}
const int N = 1010;
int n, a[N], f[N], g[N];
int main()
{
  cin >> n;
  for (int i = 1; i <= n; i++)
  cin >> a[i];
  for (int i = 1; i <= n; i++)
  {
  f[i] = 1;
  for (int j = 1; j < i; j++)
    if (a[i] > a[j])
      f[i] = max(f[i], f[j] + 1);
  }
  for (int i = n; i >= 1; i--)
  {
  g[i] = 1;
  for (int j = n; j > i; j--)
    if (a[i] > a[j])
      g[i] = max(g[i], g[j] + 1);
  }
  int ans = 0;
  for (int i = 1; i <= n; i++)
  ans = max(ans, g[i] + f[i] - 1);
  cout << ans << '\n';
  return 0;
}
\end{lstlisting}
\subsection{MSIS}
MSIS means Maximum Sum Increasing Subsequence
\begin{lstlisting}
const int N = 1010;
int n, w[N], f[N];
int main()
{
  cin >> n;
  for (int i = 0; i < n; i++) cin >> w[i];
  int res = 0;
  for (int i = 0; i < n; i++)
  {
      f[i] = w[i];
      for (int j = 0; j < i; j++)
          if (w[i] > w[j])
              f[i] = max(f[i], f[j] + w[i]);
      res = max(res, f[i]);
  }
  cout << res;
}
\end{lstlisting}
\subsection{LCIS}
LCIS means Longest Common Increasing Subsequence
\begin{lstlisting}
const int N = 3010;
int n, a[N], b[N], f[N][N];
int main()
{
  cin >> n;
  for (int i = 1; i <= n; i++)
      cin >> a[i];
  for (int i = 1; i <= n; i++)
      cin >> b[i];
  for (int i = 1; i <= n; i++)
  {
      int maxv = 1;
      for (int j = 1; j <= n; j++)
      {
          f[i][j] = f[i - 1][j];
          if (a[i] == b[j])
              f[i][j] = max(f[i][j], maxv);
          if (a[i] > b[j])
              maxv = max(maxv, f[i - 1][j] + 1);
      }
  }
  int res = 0;
  for (int i = 1; i <= n; i++)
      res = max(res, f[n][i]);
  cout << res;
}
\end{lstlisting}
\section{Knapsack Problem}
\subsection{Multiple Knapsack Problem}
\begin{lstlisting}
const int N = 20010;
int n, m, f[N], g[N], q[N];
int main()
{
  cin >> n >> m;
  for (int i = 0; i < n; i++)
  {
      int v, w, s;
      cin >> v >> w >> s;
      memcpy(g, f, sizeof f);
      for (int j = 0; j < v; j++)
      {
          int hh = 0, tt = -1;
          for (int k = j; k <= m; k += v)
          {
              if (hh <= tt && q[hh] < k - s * v)
                  hh++;
              while (hh <= tt && g[q[tt]] - (q[tt] - j) / v * w <= g[k] - (k - j) / v * w)
                  tt--;
              q[++tt] = k;
              f[k] = g[q[hh]] + (k - q[hh]) / v * w;
          }
      }
  }
  cout << f[m] << '\n';
}
\end{lstlisting}
\subsection{Two-Dimensional Cost Knapsack Problem}
\begin{lstlisting}
const int N = 110;
int n, V, M, f[N][N];
int main()
{
  cin >> n >> V >> M;
  for (int i = 0; i < n; i++)
  {
      int v, m, w;
      cin >> v >> m >> w;
      for (int j = V; j >= v; j--)
          for (int k = M; k >= m; k--)
              f[j][k] = max(f[j][k], f[j - v][k - m] + w);
  }
  cout << f[V][M] << '\n';
}
\end{lstlisting}
\subsection{Finding the Actual Solution Set}
\begin{lstlisting}
const int N = 1010;
int n, m;
int v[N], w[N], f[N][N];
int main()
{
  cin >> n >> m;
  for (int i = 1; i <= n; i++)
      cin >> v[i] >> w[i];
  for (int i = n; i >= 1; i--)
      for (int j = 0; j <= m; j++)
      {
          f[i][j] = f[i + 1][j];
          if (j >= v[i])
              f[i][j] = max(f[i][j], f[i + 1][j - v[i]] + w[i]);
      }
  int j = m;
  for (int i = 1; i <= n; i++)
      if (j >= v[i] && f[i][j] == f[i + 1][j - v[i]] + w[i])
      {
          cout << i << ' ';
          j -= v[i];
      }
}
\end{lstlisting}
\subsection{Maximum Linearly Independent Subset}
\begin{lstlisting}
const int N = 110, M = 25010;
int n, v[N];
bool f[M];
int main()
{
  int T; cin >> T;
  while (T--)
  {
      cin >> n;
      for (int i = 1; i <= n; ++i)
          cin >> v[i];
      sort(v + 1, v + n + 1);
      int m = v[n], res = 0;
      memset(f, 0, sizeof f);
      f[0] = true; // 状态的初值
      for (int i = 1; i <= n; ++i)
      {
          if (f[v[i]]) continue;
          res++;
          for (int j = v[i]; j <= m; ++j)
              f[j] |= f[j - v[i]];
      }
      cout << res << '\n';
  }
}
\end{lstlisting}
\subsection{Mixed Knapsack Problem}
\begin{lstlisting}
const int N = 1010;
int n, m, f[N];
int main()
{
  cin >> n >> m;
  for (int i = 0; i < n; i++)
  {
      int v, w, s;
      cin >> v >> w >> s;
      if (!s)
      {
          for (int j = v; j <= m; j++)
              f[j] = max(f[j], f[j - v] + w);
      }
      else
      {
          if (s == -1)
              s = 1;
          for (int k = 1; k <= s; k *= 2)
          {
              for (int j = m; j >= k * v; j--)
                  f[j] = max(f[j], f[j - k * v] + k * w);
              s -= k;
          }
          if (s)
          {
              for (int j = m; j >= s * v; j--)
                  f[j] = max(f[j], f[j - s * v] + s * w);
          }
      }
  }
  cout << f[m] << '\n';
}
\end{lstlisting}
\subsection{Dependent Knapsack Problem}
\begin{lstlisting}
const int N = 110;
int n, m, root;
int h[N], e[N], ne[N], idx;
int v[N], w[N], [N][N];
void add(int a, int b)
{
  e[idx] = b, ne[idx] = h[a], h[a] = idx++;
}
void dfs(int u)
{
  for (int i = h[u]; ~i; i = ne[i])
  {
      int son = e[i];
      dfs(son);
      for (int j = m - v[u]; j >= 0; --j)
          for (int k = 0; k <= j; ++k)
              f[u][j] = max(f[u][j], f[u][j - k] + f[son][k]);
  }
  for (int j = m; j >= v[u]; --j)
      f[u][j] = f[u][j - v[u]] + w[u];
  for (int j = 0; j < v[u]; ++j)
      f[u][j] = 0;
}
int main()
{
  memset(h, -1, sizeof h);
  cin >> n >> m;
  for (int i = 1; i <= n; ++i)
  {
      int p;
      cin >> v[i] >> w[i] >> p;
      if (p == -1) root = i;
      else add(p, i);
  }
  dfs(root);
  cout << f[root][m] << '\n';
}
\end{lstlisting}
\subsection{Number of Solutions}
\begin{lstlisting}
const int N = 1010, mod = 1e9 + 7;
int n, m;
int w[N], v[N], f[N], g[N];
int main()
{
  cin >> n >> m;
  for (int i = 1; i <= n; ++i)
      cin >> v[i] >> w[i];
  g[0] = 1;
  for (int i = 1; i <= n; ++i)
  {
      for (int j = m; j >= v[i]; --j)
      {
          int temp = max(f[j], f[j - v[i]] + w[i]), c = 0;
          if (temp == f[j])
              c = (c + g[j]) % mod;
          if (temp == f[j - v[i]] + w[i])
              c = (c + g[j - v[i]]) % mod;
          f[j] = temp, g[j] = c;
      }
  }
  int res = 0;
  for (int j = 0; j <= m; ++j)
      if (f[j] == f[m])
          res = (res + g[j]) % mod;
  cout << res << '\n';
}
\end{lstlisting}
\section{FSM}
\begin{lstlisting}
const int N = 100010;
int n, w[N], f[N][2];
int main()
{
  int T; cin >> T;
  while (T--)
  {
      cin >> n;
      for (int i = 1; i <= n; i++)
          cin >> w[i];
      for (int i = 1; i <= n; i++)
      {
          // YOUR_FSM_RULES
          // f[i][0] =
          // f[i][1] =
      }
      cout << max(f[n][0], f[n][1]) << '\n';
  }
}
\end{lstlisting}
\section{Digit DP}
\begin{lstlisting}
const int N = 35;
int l, r, k, b, a[N], al, f[N][N];
int dp(int pos, int st, int op)
{
  if (!pos) return st == k;
  if (!op && ~f[pos][st])
      return f[pos][st];
  int res = 0, maxx = op ? min(a[pos], 1) : 1;
  for (int i = 0; i <= maxx; i++)
  {
      if (st + i > k) continue;
      res += dp(pos - 1, st + i, op && i == a[pos]);
  }
  return op ? res : f[pos][st] = res;
}
int calc(int x)
{
  al = 0;
  memset(f, -1, sizeof f);
  while (x) a[++al] = x % b, x /= b;
  return dp(al, 0, 1);
}
int main()
{
  cin >> l >> r >> k >> b;
  cout << calc(r) - calc(l - 1) << '\n';
}
\end{lstlisting}
\section{Queue Optimization for DP}
\begin{lstlisting}
int n, m, s[300010], q[300010];
int main()
{
  cin >> n >> m;
  for (int i = 1; i <= n; i++)
      cin >> s[i], s[i] += s[i - 1];
  int res = INT_MIN, hh = 0, tt = 0;
  for (int i = 1; i <= n; i++)
  {
      if (q[hh] < i - m) hh++;
      res = max(res, s[i] - s[q[hh]]);
      while (hh <= tt && s[q[tt]] >= s[i]) tt--;
      q[++tt] = i;
  }
}
\end{lstlisting}
